%%
%% Package: spectralsequences v1.1.0-dev
%% Author: Hood Chatham
%% Email: hood@mit.edu
%% Date: 2017-09-07
%% License: Latex Project Public License
%%
%% File: sseqparsers.code.tex
%%
%%    Defines the parsers and input sanitizers. Also handles tikz primitive parsing and \isalive.
%%    ExplSyntaxOn for this whole file!

\ExplSyntaxOn
\let\sseq@NoValue\c__xparse_no_value_tl

\protected\def\sseqparseint#1#2{
    \sseq@ifintexpr{#2+0}{ % +0 here to prevent \lastx from freaking out if it's at the end of #2.
        \edef#1{\the\numexpr#2\relax}
    }{
        \sseq@error{invalid-int-expr}
    }
}

% WARNING: doesn't work =(
\def\sseq@ifint#1{
    \ifcat$\romannumeral0#1$
        \prg_return_true:
    \else
        \ifcat$\romannumeral-0#1$
            \prg_return_true:
        \else
            \prg_return_false:
        \fi
    \fi
}

% Prevent \protect from causing trouble for \pgfmathparse
\let\sseq@pgfmathparse@orig\pgfmathparse
\protected\def\sseq@pgfmathparse#1{\let\@@protect\protect\def\protect{}\sseq@pgfmathparse@orig{#1}\restore@protect}


%%% ifpgfmathexpr
% #1 -- expresion to test
% #2 -- true case
% #3 -- false case
% This tests true if \pgfmathparse{#1} throws an error or not. Luckily, \pgfmathparse pipes its errors through \pgfmath@error so this is easy.
\def\sseq@ifpgfmathexpr#1{
    \begingroup
    \let\protect\pgfutil@empty
    \global\sseq@gtempiftrue
    \def\pgfmath@error##1##2{\global\sseq@gtempiffalse\sseq@closegroups\sseq@break}
    \def\sseq@groupflag{}
    \pgfmathparse{#1}
    \global\let\ifpgfmathunitsdeclared\ifpgfmathunitsdeclared
    \ifpgfmathunitsdeclared
        \xdef\sseq@mathresult{\pgfmathresult pt}
    \else
        \xdef\sseq@mathresult{\pgfmathresult}
    \fi
    \endgroup
    \sseq@breakpoint
    \ifsseq@gtempif
        \@xp\@firstoftwo
    \else
        \@xp\@secondoftwo
    \fi
}

\def\sseq@closegroups{\ifx\sseq@groupflag\empty\endgroup\@xp\sseq@closegroups\fi}


% Test if single token input is a digit
\def\sseq@ifdigit#1{
    \ifodd0
        \ifx#1 0 1 \else
        \ifx#1 1 1 \else
        \ifx#1 2 1 \else
        \ifx#1 3 1 \else
        \ifx#1 4 1 \else
        \ifx#1 5 1 \else
        \ifx#1 6 1 \else
        \ifx#1 7 1 \else
        \ifx#1 8 1 \else
        \ifx#1 9 1 \else
        \fi \fi \fi \fi \fi
        \fi \fi \fi \fi \fi
    \relax
        \@xp\@firstoftwo
    \else
        \@xp\@secondoftwo
    \fi
}

%%% ifintexpr
% #1 -- expresion to test
% #2 -- true case
% #3 -- false case
% This tests true if \numexpr #1\relax throws no error and consumes all of #1 and false otherwise.

% Spaces matter to us, because \sseq@ifintexpr{1 1} is false but \sseqsifintexpr{11} is true
% so we make spaces into catcode other and use \scantokens
\def\sseq@ifintexpr#1{%
    \bgroup
    \sseq@tempcount\z@
    \sseq@ifintexpr@{needsint}#1\sseq@nil%
    \@xp\egroup\next
}

% We need to use \futurelet so that we can detect open braces even when they only surround one token like {1}
% also we use it to detect spaces. Store the state in \sseq@ifintexpr@state first.
\def\sseq@ifintexpr@#1{\def\sseq@ifintexpr@state{#1}\futurelet\testtok\sseq@ifintexpr@@}
\def\sseq@ifintexpr@@{%
    \ifx\testtok\bgroup%
        \let\next\sseq@ifintexpr@false
    \else
        \ifx\testtok\sseq@nil
            \@xp\let\@xp\next\csname sseq@ifintexpr@@\sseq@ifintexpr@state @done\endcsname
        \else
            % We need to check here for a space because \string<space> produces NO OUTPUT regardless of the catcode of the space.
            % This messes up \sseq@ifintexpr@@@ because it doesn't expect \string#1 to produce no characters.
            \@xp\ifx\space\testtok
                \def\next{\sseq@ifintexpr@next{\space}\@xp\next\romannumeral-`0}
            \else
                \let\next\sseq@ifintexpr@@@
            \fi
        \fi
    \fi
    \next
}

\def\sseq@ifintexpr@@@#1{
    \ifcat$\@xp\@gobble\string#1$%
        \@xp\sseq@ifintexpr@@@@\@xp#1
    \else
        % This is a macro, so fexpand it
        % Then use f expansion.
        \@xp\sseq@ifintexpr@@@fexpcs\@xp#1
    \fi
}

\def\sseq@ifintexpr@@@fexpcs{\exp_last_unbraced:Nf\sseq@ifintexpr@@@fexpcs@}
\def\sseq@ifintexpr@@@fexpcs@{\futurelet\testtok\sseq@ifintexpr@@@fexpcs@@}
\def\sseq@ifintexpr@@@fexpcs@@{
    \ifx\testtok\bgroup
        \@xp\sseq@ifintexpr@false % We already tested for groups above, so we need to check if this expanded to a group
    \else
        \@xp\sseq@ifintexpr@@@@ % If it's still a control sequence, then this will fail in the \pgfutil@ifundefined step
    \fi
}



% We can't just use \futurelet because "\let\testtok(" makes \testtok unexpandable
% (I guess that makes sense, but why is it that I need \@xp\ifx\otherspace above if I've also \let\otherspace to a character? Mysterious...),
% so then "\csname hello\testtok\endcsname" is an error. This indexes into our state machine,
% cases: a digits, + or -, * or /, (, ), or something else (anything else always leads to false
\def\sseq@ifintexpr@@@@#1{%
    \ifx#1\sseq@nil
        \def\next{\csname sseq@ifintexpr@@\sseq@ifintexpr@state @done\endcsname\sseq@nil}%
    \else
        \sseq@ifdigit{#1}%
            {\sseq@ifintexpr@next{digit}}%
            {%
                \ifx#1+%
                    \sseq@ifintexpr@next{+-}
                \else
                    \ifx#1-%
                        \sseq@ifintexpr@next{+-}
                    \else
                        \ifx#1*%
                            \sseq@ifintexpr@next{*/}
                        \else
                            \ifx#1/%
                                \sseq@ifintexpr@next{*/}
                            \else
                                % This extra \string here is so that if a control sequence fexpanded and still gave a control sequence,
                                % we don't get a missing \endcsname error here, it just returns false
                                \pgfutil@ifundefined{sseq@ifintexpr@@\sseq@ifintexpr@state @\string#1}%
                                    {\let\next\sseq@ifintexpr@false}%
                                    {\sseq@ifintexpr@next{#1}}%
                            \fi
                        \fi
                    \fi
                \fi
            }%
    \fi
    \next
}

\def\sseq@ifintexpr@true#1\sseq@nil{\ifnum\sseq@tempcount=\z@ \let\next\@firstoftwo\else\let\next\@secondoftwo\fi}
\def\sseq@ifintexpr@false#1\sseq@nil{\let\next\@secondoftwo}

\def\sseq@makeifint#1#2#3{\@xp\def\csname sseq@ifintexpr@@#1@#2\endcsname{#3}}
\def\sseq@ifintexpr@next#1{\@xp\let\@xp\next\csname sseq@ifintexpr@@\sseq@ifintexpr@state @#1\endcsname}

\sseq@makeifint{needsint}{done}{\sseq@ifintexpr@false}
\sseq@makeifint{needsint}{digit}{\sseq@ifintexpr@{int}}
\sseq@makeifint{needsint}{*/}{\sseq@ifintexpr@false}
\sseq@makeifint{needsint}{+-}{\sseq@ifintexpr@{needsint}}
\sseq@makeifint{needsint}{(}{\advance\sseq@tempcount\@ne\sseq@ifintexpr@{needsint}}
\sseq@makeifint{needsint}{)}{\sseq@ifintexpr@false}
\sseq@makeifint{needsint}{\space}{\sseq@ifintexpr@{needsint}}

\sseq@makeifint{int}{done}{\sseq@ifintexpr@true}
\sseq@makeifint{int}{digit}{\sseq@ifintexpr@{int}}
\sseq@makeifint{int}{*/}{\sseq@ifintexpr@{needsint}}
\sseq@makeifint{int}{+-}{\sseq@ifintexpr@{needsint}}
\sseq@makeifint{int}{(}{\sseq@ifintexpr@false}
\sseq@makeifint{int}{)}{
    \advance\sseq@tempcount\m@ne
    \ifnum\sseq@tempcount<\z@\relax
        \@xp\@xp\@xp\sseq@ifintexpr@false\@xp\@gobble
    \else
        \@xp\sseq@ifintexpr@
    \fi{nointallowed}
}
\sseq@makeifint{int}{\space}{\sseq@ifintexpr@{nointallowed}}

\sseq@makeifint{nointallowed}{done}{\sseq@ifintexpr@true}
\sseq@makeifint{nointallowed}{digit}{\sseq@ifintexpr@false}
\sseq@makeifint{nointallowed}{*/}{\sseq@ifintexpr@{needsint}}
\sseq@makeifint{nointallowed}{+-}{\sseq@ifintexpr@{needsint}}
\sseq@makeifint{nointallowed}{(}{\sseq@ifintexpr@false}
\sseq@makeifint{nointallowed}{)}{
    \advance\sseq@tempcount\m@ne
    \ifnum\sseq@tempcount<\z@
        \@xp\@xp\@xp\sseq@ifintexpr@false\@xp\@gobble
    \else
        \@xp\sseq@ifintexpr@
    \fi{nointallowed}
}
\sseq@makeifint{nointallowed}{\space}{\sseq@ifintexpr@{nointallowed}}

%%
%% This parser defines the syntax for the page argument of \d.
%%
\def\sseq@d@grabpage#1{
    \let\sseq@grabdpage@return#1
    \futurelet\testtok\sseq@grabdpage@
}

\def\sseq@grabdpage@{
    \sseq@tempiftrue
    \ifx\testtok\bgroup
        \let\next\sseq@grabdpage@group
    \else
        \def\sseq@dpage{}
        \let\next\sseq@grabdpage@norm
    \fi
    \next
}

\def\sseq@grabdpage@group#1{\def\sseq@dpage{#1}\sseq@grabdpage@return}

\def\sseq@grabdpage@norm{
    \futurelet\testtok\sseq@grabdpage@norm@
}

\def\sseq@grabdpage@norm@{
    \let\next\sseq@grabdpage@token
    \ifx\testtok\bgroup
        \sseq@tempiffalse
        \let\next\sseq@grabdpage@return
    \else
        \ifx\testtok\par
            \sseq@tempiffalse
            \let\next\sseq@grabdpage@return
        \else
            \ifx\testtok(
                \let\next\sseq@grabdpage@return
            \else
                \@xp\ifx\space\testtok
                    \let\next\sseq@grabdpage@space
                \fi
            \fi
        \fi
    \fi
    \next
}

\def\sseq@grabdpage@space{\sseq@d@addto@macro\sseq@dpage{~}\@xp\sseq@grabdpage@norm\romannumeral-`0}

\def\sseq@grabdpage@token#1{
    \ifcat$\@xp\@gobble\string#1$%
        \sseq@d@addto@macro\sseq@dpage{#1}
        \@xp\sseq@grabdpage@norm
    \else
        \ifx#1\end
            \sseq@tempiffalse
            \@xp\sseq@grabdpage@return\@xp#1\romannumeral-`0
        \else
            \ifx#1\begin
                \sseq@tempiffalse
                \@xp\sseq@grabdpage@return\@xp#1\romannumeral-`0
            \else
                \@xp\sseq@grabdpage@token@fexpcs\@xp#1\romannumeral-`0
            \fi
        \fi
    \fi
}

\def\sseq@grabdpage@token@fexpcs{\exp_last_unbraced:Nf\sseq@grabdpage@token@fexpcs@}
\def\sseq@grabdpage@token@fexpcs@{\futurelet\testtok\sseq@grabdpage@token@fexpcs@@}
\def\sseq@grabdpage@token@fexpcs@@{
    \ifx\testtok\bgroup
        \sseq@tempiffalse
        \@xp\sseq@grabdpage@return
    \else
        \@xp\sseq@grabdpage@token@fexpcs@@@
    \fi
}
\def\sseq@grabdpage@token@fexpcs@@@#1{
    \ifcat$\@xp\@gobble\string#1$%
        \sseq@d@addto@macro\sseq@dpage{#1}
        \@xp\sseq@grabdpage@norm
    \else
        \sseq@tempiffalse
        \@xp\sseq@grabdpage@return
    \fi
}


%%
%%
%%          New Class Pattern
%%
%%

\newcount\sseq@cp@row
\newcount\sseq@cp@n

\def\sseqnewclasspattern#1#2{
    \ifcsname sseq@#1xoffset1/1\endcsname\sseq@error@n{classpattern-already-defined}{#1}\fi % could be a warning
    \begingroup
    \def\sseq@name{#1}
    \def\sseq@temp{}
    \def\sseq@state{nocoord}
    \sseq@cp@row=\@ne
    \sseq@cp@n=\z@
    \let\next\sseq@newclasspattern@
    \sseq@newclasspattern@#2\sseq@nil
}

\def\sseq@newclasspattern@{
    \@ifnextchar\bgroup{\sseq@error@n{classpattern-unexpected-token}{\{}}{\sseq@newclasspattern@processstate}
}

\def\sseq@newclasspattern@processstate{
    \csname sseq@newclasspattern@processstate@\sseq@state\endcsname
}

\def\sseq@newclasspattern@processstate@xcoord#1{
    \ifx#1,
        \sseq@ifpgfmathexpr{\sseq@temp}{
            \let\sseq@tempx\sseq@mathresult
            \def\sseq@temp{}
            \def\sseq@state{ycoord}
        }{
            \sseq@error{classpattern-not-integer-expression}
            \let\next\sseq@newclasspattern@abort
        }
    \else
        \ifx#1\sseq@nil
            \sseq@error{classpattern-missing-tokens}
            \let\next\endgroup
        \else
            \sseq@d@addto@macro\sseq@temp{#1}
        \fi
    \fi
    \next
}

\def\sseq@newclasspattern@processstate@ycoord#1{
    \ifx#1)
        \sseq@ifpgfmathexpr{\sseq@temp}{% Need iffloatexpr =(
            \let\sseq@tempy\sseq@mathresult
            \def\sseq@temp{}
            \def\sseq@state{nocoord}
            \advance\sseq@cp@n\@ne
            \ifnum\sseq@cp@n>\sseq@cp@row\relax
                \sseq@error@x{classpattern-extra-coord-ignored}{(\sseq@tempx,\sseq@tempy)}
            \else
                \sseq@e@addto@macro\sseq@newclasspattern@dodefs{
                    \def\@xp\@nx\csname sseq@\sseq@name xoffset\the\sseq@cp@n/\the\sseq@cp@row\endcsname{\sseq@tempx}
                    \def\@xp\@nx\csname sseq@\sseq@name yoffset\the\sseq@cp@n/\the\sseq@cp@row\endcsname{\sseq@tempy}
                }
            \fi
        }{
            \sseq@error{classpattern-not-integer-expression}
            \let\next\sseq@newclasspattern@abort
        }
    \else
        \ifx#1\sseq@nil
            \sseq@error{classpattern-missing-tokens}
            \let\next\endgroup
        \else
            \sseq@d@addto@macro\sseq@temp{#1}
        \fi
    \fi
    \next
}

\def\sseq@newclasspattern@processstate@nocoord#1{
    \ifx#1(
        \def\sseq@state{xcoord}
    \else
        \ifx#1;
            \ifnum\sseq@cp@n=\sseq@cp@row\relax
                \advance\sseq@cp@row\@ne
                \sseq@cp@n=\z@
            \else
                \sseq@error@x{classpattern-too-few-coords}{\the\sseq@cp@row}
                \let\next\sseq@newclasspattern@abort
            \fi
        \else
            \ifx#1\sseq@nil
                \let\next\sseq@newclasspattern@finish
            \else
                \sseq@error@n{classpattern-unexpected-token}{#1}
                \let\next\sseq@newclasspattern@abort
            \fi
        \fi
    \fi
    \next
}

\def\sseq@newclasspattern@finish{
    \ifnum\sseq@cp@n=\z@\relax
        \advance\sseq@cp@row\m@ne
    \else
        \ifnum\sseq@cp@n=\sseq@cp@row\relax\else
            \sseq@error@x{classpattern-too-few-coords}{\the\sseq@cp@row}
            \let\sseq@newclasspattern@dodefs\empty
        \fi
    \fi
    % This will make a definition even if the class pattern was rejected because of the too-few-coords error
    % It doesn't matter though, because we don't use this to check for existence
    \sseq@e@addto@macro\sseq@newclasspattern@dodefs{%
        \chardef\@xp\@nx\csname sseq@\sseq@name @maxclasses\endcsname=\the\sseq@cp@row\relax
    }%
    \@xp\endgroup
    \sseq@newclasspattern@dodefs


}
\def\sseq@newclasspattern@dodefs{}
\def\sseq@newclasspattern@abort#1\sseq@nil{\endgroup}




%%%%%%                                            %%%%%%
%%                                                    %%
%%               Coordinate Parsers                   %%
%%                                                    %%
%%%%%%                                            %%%%%%


\def\sseq@ifdead#1{\ifnum\sseq@obj{class.#1[\sseq@obj{class.#1.num}].page}<\sseq@infinitycount\@xp\@firstoftwo\else\@xp\@secondoftwo\fi}


% These do all of the work of taking a coordinate of the form (x,y), (x,y,n), or (x,y,tag) and turning them
% into the internal representation needed for the rest of the package.
% This comes in two parts:
%   \sseq@parsecoord@maincoord -- separates a coordinate into {x,y}{n or tag}, numerically evaluates x and y, and adds in the values of \sseq@x and \sseq@y
%   \sseq@parsecoord@index -- decides which class {n or tag} represents
%
% There are three interface commands:
%    \sseq@parsecoord   -- for most things
%    \sseq@cparsecoord  -- for \class (but not for \replaceclass or \classoptions). Doesn't use \sseq@parsecoord@index at all.
%    \sseq@dparsecoord  -- for \d and \doptions
%    \sseq@parsecoordex -- for \isalive: the body of a page constraint needs to be \edef'd so we need a completely expandable version.
%      The expandable version doesn't have any error checking -- to get the error checking, we pass the coordinates once through the unexpandable version
%      inside of an hbox.

\def\sseq@errortype{}
\def\sseq@printerrortype#1{\@xp\sseq@ifempty\@xp{\sseq@errortype}{}{#1 \sseq@errortype}}
\def\sseq@printforerrortype{\sseq@printerrortype{~for~}}

%% Pulls off x,y and evaluates them, puts n or empty into a separate variable.
% #1 -- a coordinate of the form {x,y} or {x,y,stuff}
\let\sseq@relax\relax
% The \sseq@relax is just to make \lastclass work. It has to pull an argument off the token stream to expandably check whether
% it is an integer. However, without this \sseq@relax, \lastclass could appear as the last token of the edef, which is no good.
\def\sseq@parsecoord@maincoord#1{\exp_last_unbraced:Nx\sseq@parsecoord@maincoord@{#1\sseq@relax}}
\def\sseq@parsecoord@maincoord@#1\sseq@relax{%
    \pgfutil@in@ \sseq@protecterror { #1 }
    \ifpgfutil@in@
        \bgroup
        \sseq@restorefont % This prevents "Missing character" stuff from being written to the log.
        \let\sseq@protecterror\relax
        \setbox0=\hbox{#1} % Trigger the error message
        \egroup
        \@xp\sseq@break
    \else
        \pgfutil@in@,{#1}
        \ifpgfutil@in@
            \sseq@parsecoord@maincoord@@#1,\sseq@nil
        \else
            \sseq@obj@ifdef{class.namedclass.\detokenize{#1}}{
                \def\sourcename{#1}
                \@xpthree\sseq@parsecoord@maincoord@named\sseq@obj{class.namedclass.#1}\sseq@nil
            }{
                \sseq@error{invalid-coordinate}\sseq@breakfifi
            }
        \fi
    \fi
}

\def\sseq@parsecoord@maincoord@@#1,#2,#3\sseq@nil{%
    \sseq@obj@ifdef{class.namedclass.\detokenize{#1}}{
        \def\sourcename{#1}
        \@xpthree\sseq@parsecoord@maincoord@named\sseq@obj{class.namedclass.#1}\sseq@nil
        \edef\sseq@restcoord{\sseq@removecomma#2,#3\sseq@nil}
    }{
        \sseq@ifintexpr{#1}{
            \sseq@ifintexpr{#2}{}{\sseq@error@n{invalid-coordinate}{y~}\sseq@breakfifi} % breakfifi to get out of ifs from maincoord@
            \edef\sseq@xcoord{\the\numexpr\sseq@x+(#1)\relax}%
            \edef\sseq@ycoord{\the\numexpr\sseq@y+(#2)\relax}%
            \edef\sseq@xycoord{\sseq@xcoord,\sseq@ycoord}%
            \sseq@ifempty{#3}{\let\sseq@restcoord\pgfutil@empty}{%
                \edef\sseq@restcoord{\sseq@removecomma#3\sseq@nil}%
            }%
        }{
            \sseq@error@n{invalid-coordinate}{x~}\sseq@breakfifi % breakfifi to get out of ifs from maincoord@
        }
    }%
}%

\def\sseq@parsecoord@maincoord@named#1,#2,#3\sseq@nil{
    \def\sseq@xcoord{#1}%
    \def\sseq@ycoord{#2}%
    \edef\sseq@xycoord{\sseq@xcoord,\sseq@ycoord}
    \def\sseq@restcoord{}
    \pgfkeys@spdef\sseq@rawindex{#3}
    \def\sseq@index{#3}
    \let\rawindex\sseq@rawindex
    \let\index\sseq@index
}

\def\sseq@removecomma#1,\sseq@nil{#1}


% "Input" to this command is stored in the following commands:
% \sseq@xycoord -- x,y
% \sseq@rawindex -- n
% \sseq@errortype -- description for error messages
\def\sseq@errortype@tikzprim{tikz ~ primitive}% This is sort of a silly hack, but this first error check shouldn't happen in a tikz primitive.
\def\sseq@parsecoord@index{%
    \sseq@obj@ifundef{partcoord.(\sseq@xycoord).numnodes}{
        \ifx\sseq@errortype\sseq@errortype@tikzprim
            \def\sseq@index{}%  Communicates to \sseq@tikzprimitives@coords@maybeclass that this is not a class
        \else
            \sseq@error@x{no-classes}{(\sseq@xycoord)}\@xp\sseq@break
        \fi
    }{
        \ifx\sseq@rawindex\pgfutil@empty
            \def\sseq@index{1}
        \else
            \sseq@obj@ifdef{partcoord.(\sseq@xycoord).tag.\sseq@class@tagprefix\sseq@rawindex}{% If the raw index is a tag
                \edef\sseq@index{\sseq@obj{partcoord.(\sseq@xycoord).tag.\sseq@class@tagprefix\sseq@rawindex}}
            }{%
                \sseq@ifintexpr{\sseq@rawindex}{%
                    \edef\sseq@rawindexeval{\the\numexpr\sseq@rawindex}
                    \ifnum\sseq@rawindexeval<\z@%
                        \edef\sseq@index{\the\numexpr\sseq@obj{partcoord.(\sseq@xycoord).numnodes} + \sseq@rawindex + \@ne\relax}%
                        \ifnum\sseq@index<\@ne
                            \sseq@error@xxx{index-too-large}{\sseq@rawindex}{\sseq@obj{partcoord.(\sseq@xycoord).numnodes}}{(\sseq@xycoord)}%
                            \edef\sseq@index{1}% or maybe I should break?
                        \fi
                    \else
                        \ifnum\sseq@rawindexeval>\sseq@obj{partcoord.(\sseq@xycoord).numnodes}\relax
                            \sseq@error@xxx{index-too-large}{\sseq@rawindex}{\sseq@obj{partcoord.(\sseq@xycoord).numnodes}}{(\sseq@xycoord)}%
                            \edef\sseq@rawindex{\sseq@obj{partcoord.(\sseq@xycoord).numnodes}}% or maybe I should break?
                        \fi
                        \edef\sseq@index{\the\numexpr\sseq@rawindex}
                    \fi
                }{
                    \sseq@error@xx{undefined-tag}{\sseq@rawindex}{(\sseq@xycoord)}%
                    \def\sseq@index{1}%
                }
            }
        \fi
    }
}

% #1 -- name to define
% #2 -- coordinate
% #3 -- the error message
\def\sseq@parsecoord#1#2#3{%
    \def\sseq@index{}
    \def\sseq@errortype{#3}
    \sseq@parsecoord@maincoord{#2}%
    \let\sseq@rawindex\sseq@restcoord
    \ifx\sseq@index\pgfutil@empty
        \@xp\sseq@parsecoord@index
    \fi
    % TODO: clean this up a bit? it's pretty ad-hoc.
    % Whenever you update this, make sure to update sseq@paredcoords@swap below too.
    \@xp\let\csname x\sseq@macroname#1\endcsname\sseq@xcoord
    \@xp\let\csname y\sseq@macroname#1\endcsname\sseq@ycoord
    \@xp\let\csname n\sseq@macroname#1\endcsname\sseq@index
    \@xp\let\csname \sseq@macroname#1@partial\endcsname\sseq@xycoord
    \edef#1{(\sseq@xycoord,\sseq@index)}%
    \@xp\edef\csname\sseq@macroname#1nopar\endcsname{\sseq@xycoord,\sseq@index}
    \@xp\def\csname\sseq@macroname#1name\endcsname{(#2)}%
    \@xp\def\csname\sseq@macroname#1nameunbraced\endcsname{#2}%
    \@xp\edef\csname\sseq@macroname#1@internalname\endcsname{sseq{\sseq@xycoord,\sseq@index}}%
}

% #1 & #2 -- the two base commands that were handed to \sseq@parsecoord to swap.
% This is only used in \structline to normalize the name of structure lines, so that sseqpages knows what you are talking about
% if you say \structline(coorda)(coordb) and then later \structline(coordb)(coorda)
\cs_new:Npn \sseq_swapmacros:NN #1#2 {
    \let\sseq@temp#1 \let#1#2 \let#2\sseq@temp
}
\cs_generate_variant:Nn \sseq_swapmacros:NN { cc }

\def\sseq@parsedcoords@swap#1#2{
    \sseq_swapmacros:cc { x\sseq@macroname#1 } { x\sseq@macroname#2 }
    \sseq_swapmacros:cc { y\sseq@macroname#1 } { y\sseq@macroname#2 }
    \sseq_swapmacros:cc { n\sseq@macroname#1 } { n\sseq@macroname#2 }
    \sseq_swapmacros:cc { \sseq@macroname#1@partial } { \sseq@macroname#2@partial }
    \sseq_swapmacros:NN #1 #2
    \sseq_swapmacros:cc { \sseq@macroname#1nopar } { \sseq@macroname#2nopar }
    \sseq_swapmacros:cc { \sseq@macroname#1name } { \sseq@macroname#2name }
    \sseq_swapmacros:cc { \sseq@macroname#1nameunbraced } { \sseq@macroname#2nameunbraced }
    \sseq_swapmacros:cc { \sseq@macroname#1@internalname } { \sseq@macroname#2@internalname }
}

% EXPOSE: a wrapper around \sseq@parsecoord. Use full name of coordinate to avoid name class with sseq@parsecoord.
\sseq@DeclareDocumentCommand \parsecoordinate { m r() }{%
    \sseq@parsecoord{#1}{#2}{call of \string\parsecoord}%
}

% #1 -- coordinate
\def\sseq@cparsecoord#1{%
    \def\sseq@index{}
    \sseq@parsecoord@maincoord{#1}%
    \ifx\sseq@index\pgfutil@empty\else\sseq@error{named-coordinate-class}\fi% probably not possible to get this error
    \ifx\sseq@restcoord\pgfutil@empty\else\sseq@error{class-extra-coords}\fi%
    \let\partialcoord\sseq@xycoord
    \let\xcoord\sseq@xcoord
    \let\ycoord\sseq@ycoord
    \edef\nodenum{\sseq@obj@ifundef{partcoord.(\partialcoord).numnodes}{1}{\the\numexpr\sseq@obj{partcoord.(\partialcoord).numnodes}+1}}%
    \edef\coordnopar{\xcoord,\ycoord,\nodenum}
    \edef\coord{(\coordnopar)}%
}


% #1 -- page
% #2 -- coordinate
% #3 -- target coordinate or "No Value"
\def\sseq@dparsecoord#1#2#3{%
    \def\sseq@index{}%
    \def\sseq@errortype{differential}%
    \sseq@parsecoord@maincoord{#2}%
    \ifx\sseq@index\pgfutil@empty
        \edef\sseq@rawsindex{\@xp\sseq@dparsecoord@getxindex\@xp{\sseq@restcoord}}%
        \@xp\pgfkeys@spdef\@xp\sseq@rawsindex\@xp{\sseq@rawsindex}
        \edef\sseq@rawtindex{\@xp\sseq@dparsecoord@getyindex\@xp{\sseq@restcoord}}%
        \@xp\pgfkeys@spdef\@xp\sseq@rawtindex\@xp{\sseq@rawtindex}
%
        \let\sseq@rawindex\sseq@rawsindex
        \def\sseq@errortype{source ~ of ~ differential}
        \@xp\sseq@parsecoord@index % the \@xp is needed so that \sseq@break inside of parsecoordindex works properly.
    \else
        \edef\sseq@rawtindex{\@xp\sseq@dparsecoord@getxindex\@xp{\sseq@restcoord}}%
        \@xp\pgfkeys@spdef\@xp\sseq@rawtindex\@xp{\sseq@rawtindex}
        \edef\sseq@shouldbeempty{\@xp\sseq@dparsecoord@getyindex\@xp{\sseq@restcoord}}
        \ifx\sseq@shouldbeempty\pgfutil@empty\else
            \sseq@error{d-named-coord-two-indexes}
        \fi
    \fi
    \let\sourcecoord\sseq@xycoord
%
    \let\xsource\sseq@xcoord
    \let\ysource\sseq@ycoord
    \let\nsource\sseq@index
%
    \edef\source{(\sseq@xycoord,\sseq@index)}%
    \ifx\sourcename\undefined
        \edef\sourcename{(\sseq@xycoord\ifx\sseq@rawindex\pgfutil@empty\else,\sseq@rawindex\fi)}
    \fi
    \IfNoValueTF{#3}{
        \ifsseq@hasdegree\else\sseq@error{d-no-degree}\@xp\sseq@break\fi
        \sseq@eval{\edef\@nx\sseq@xycoord{\@nx\sseq@differential@gettarget{#1}{\sseq@xycoord}}}
        \let\sseq@rawindex\sseq@rawtindex
        \def\sseq@errortype{target ~ of ~ differential}
        \sseq@parsecoord@index
        \let\targetcoord\sseq@xycoord
        \edef\target{(\sseq@xycoord,\sseq@index)}%
        \let\xtarget\sseq@xcoord
        \let\ytarget\sseq@ycoord
        \let\ntarget\sseq@index
        \edef\targetname{(\sseq@xycoord\ifx\sseq@rawindex\pgfutil@empty\else,\sseq@rawindex\fi)}
    }{
        \ifx\sseq@rawtindex\pgfutil@empty\else
            \sseq@error{d-target-index-target-coord}
        \fi
        \sseq@parsecoord\target{#3}{differential}
        \let\targetcoord\target@partial
        \ifsseq@hasdegree
            \ifsseq@strictdegree
                \sseq@eval{\edef\@nx\sseq@checktarget{\@nx\sseq@differential@gettarget{#1}{\sourcecoord}}}
                \ifx\sseq@checktarget\targetcoord\else
                    \sseq@error@xxx{d-wrong-degree}{(\sseq@checktarget)}{\targetname}{(\targetcoord)}
                \fi
            \fi
        \fi
    }
}

\def\sseq@differential@gettarget#1#2{\sseq@differential@gettarget@{#1}#2\sseq@nil}
\def\sseq@differential@gettarget@#1#2,#3\sseq@nil{\the\numexpr#2+\sseq@targetx{#1}\relax,\the\numexpr#3+\sseq@targety{#1}\relax}

\def\sseq@getfirsttwoelts#1,#2,#3\sseq@nil{#1,#2}
\protected\def\sseq@getdtarget#1#2#3{\sseq@eval{\edef\@nx#1{\@nx\sseq@differential@gettarget{#2}{\sseq@getfirsttwoelts#3,,\sseq@nil}}}}

% Takes a comma separated list that can be empty, have one element, or two elements, returns first element (or empty if list is empty).
\def\sseq@dparsecoord@getxindex#1{\sseq@dparsecoord@getxindex@#1,\sseq@nil}
\def\sseq@dparsecoord@getxindex@#1,#2\sseq@nil{#1}
% Takes a comma separated list that can be empty, have one element, or two elements, returns second element (or empty if less than two elements).
\def\sseq@dparsecoord@getyindex#1{\sseq@dparsecoord@getyindex@#1,\sseq@nil}
\def\sseq@dparsecoord@getyindex@#1,#2\sseq@nil{%
    \sseq@ifempty{#2}{}{%
        \sseq@removecomma#2\sseq@nil%
    }%
}

% EXPOSE: a wrapper around \sseq@parsecoord.
\sseq@DeclareDocumentCommand\parsecoordinate { m r() }{%
    \sseq@parsecoord{#1}{#2}{call of \string\parsecoord}%
}


% EXPOSE: a wrapper around \sseq@dparsecoord.
\sseq@DeclareDocumentCommand \parsedifferential { m r() d() }{%
    \sseq@dparsecoord{#1}{#2}{#3}%
}

% Expandable coordinate parsing for \isalive.
% Doesn't do any error checking -- the error checking is done by passing it through \sseq@parsecoord in a non-\edef context.
% #1 -- a callback. Must be a single command b/c it's an N in \exp_args
% #2 -- the coordinate
\cs_new:Npn \exp_last_unbraced:NNNf #1#2#3#4
  {
    \exp_after:wN #1
    \exp_after:wN #2
    \exp_after:wN #3 \tex_romannumeral:D -`0 #4
  }

\def\sseq@parsecoordex{\exp_last_unbraced:NNNf\sseq@parsecoordex@}
\def\sseq@parsecoordex@#1(#2){\sseq@parsecoordex@@#1#2,\@nil}
\def\sseq@parsecoordex@@#1#2,#3,#4\@nil{
    \ifx&#4&
        \exp_last_unbraced:NNo\sseq@parsecoord@indexex#1(\the\numexpr#2\@xp,\the\numexpr#3,1)
    \else
        \exp_args:NNo\sseq@parsecoord@indexex#1{\the\numexpr#2\@xp,\the\numexpr#3\@xp\relax\@xp}\@xp{\sseq@removecomma#4\sseq@nil}
    \fi
}


% This is \sseq@parsecoord@index pared down as much as possible -- all error checking is removed because that will happen in a
% \setbox0=\hbox{stuff} sequestered calculation that uses \sseq@parsecoord@index instead.
% #1 -- callback which takes one argument --  the final coordinate in parentheses.
% #2 -- x,y
% #3 -- n
\def\sseq@parsecoord@indexex#1#2#3{%
    \sseq@obj@ifdef{partcoord.(#2).tag.\sseq@class@tagprefix#3}{% If the raw index is a tag
        \exp_args:NNo\sseq@parsecoord@indexex@#1{\the\numexpr\sseq@obj{partcoord.(#2).tag.\sseq@class@tagprefix#3}}{#2}%
    }{%
        \ifnum\numexpr#3<0\space
            \exp_args:NNo\sseq@parsecoord@indexex@#1{\the\numexpr\sseq@obj{partcoord.(#2).numnodes} + #3 + 1}{#2}%
        \else
            \exp_args:NNo\sseq@parsecoord@indexex@#1{\the\numexpr#3}{#2}%
        \fi
    }%
}
\def\sseq@parsecoord@indexex@#1#2#3{
    #1(#3,#2)
}




%%%
%%% The Stack
%%%

% This maybe should be moved into sseqmain.code.tex...
\sseq@addtostorelist\sseq@stacktop{}
\def\sseq@stackitem#1#2{\csname sseq.\sseq@thename.stack.#1.#2\endcsname} % gobble #2 which we added in for the error message
\def\sseq@getstackentry#1#2{
    \ifnum#1<\sseq@stackdepth
        \exp_args:No\sseq@getstackentry@{\sseq@stacktop}{#1}{#2}
    \else
        \sseq@protecterror{\sseq@error@xx{index-out-of-bounds}{#1}{\the\sseq@stackdepth}}
    \fi
}

\def\sseq@getstackentry@#1#2{
    \ifnum#2=\z@\@xp\@firstoftwo\else\@xp\@secondoftwo\fi
        { \sseq@stackitem{#1} }
        { \exp_args:Nco\sseq@getstackentry@ { sseq.\sseq@thename.stack.#1.next } { \the\numexpr#2-1 } }
}


\protected\def\sseq@pushstack(#1){
    \bgroup
    \sseq@parsecoord\temp{#1}{}
    \sseq@pushstack@\temp
    \egroup
}

% Note that \class directly calls pushstack@ to avoid reparsing the coordinate
\def\sseq@pushstack@{\@xp\sseq@pushstack@@}

\def\sseq@pushstack@@(#1,#2,#3){
    \global\advance\sseq@stackdepth\@ne
    \@xp\gdef\csname sseq.\sseq@thename.stack.\the\sseq@stackdepth.lastx\endcsname{#1}
    \@xp\gdef\csname sseq.\sseq@thename.stack.\the\sseq@stackdepth.lasty\endcsname{#2}
    \@xp\xdef\csname sseq.\sseq@thename.stack.\the\sseq@stackdepth.lastclass\endcsname{lastclass.\the\sseq@stackdepth}
    \sseq@obj@gdef{class.namedclass.lastclass.\the\sseq@stackdepth}{#1,#2,#3}
    \global\@xp\let\csname sseq.\sseq@thename.stack.\the\sseq@stackdepth.next\endcsname\sseq@stacktop
    \xdef\sseq@stacktop{\the\sseq@stackdepth}
}


\newcount\sseq@userstacksavecount
\protected\def\sseq@savestack{
    \global\advance\sseq@userstacksavecount\@ne
    \sseq@savestack@name{usersave@\the\sseq@userstacksavecount}
}
\protected\def\sseq@restorestack{
    \sseq@restorestack@name{usersave@\the\sseq@userstacksavecount}
    \global\advance\sseq@userstacksavecount\m@ne
}

\def\sseq@savestack@name#1{\sseq@obj@xdef{stack.save#1}{\sseq@stacktop}}

\def\sseq@restorestack@name#1{\xdef\sseq@stacktop{\sseq@obj{stack.save#1}}}

\def\sseq@lastx#1{
    % \romannumeral is testing here whether #1 is a nonnegative integer
    % if #1 consists of a single nonnegative integer then -0#1 will be turned into some nonpositive integer
    % and romannumeral produces no output when handed a nonpositive integer, so the result will be empty.
    % If #1 has any nonnumerical tokens or is negative, there will be left over stuff.
    %
    % We also have to subtract off \sseq@x so that this behaves correctly inside of a scope (the \lastx value should be unshifted)
    \@xp\sseq@ifempty\@xp{\romannumeral-0#1}{
        \the\numexpr\sseq@getstackentry{#1}{lastx}-\sseq@x\relax
    }{
        \the\numexpr\sseq@getstackentry{0}{lastx}-\sseq@x\relax#1 % #1 isn't a number so put it back
    }
}
\def\sseq@lasty#1{
    \@xp\sseq@ifempty\@xp{\romannumeral-0#1}{
        \the\numexpr\sseq@getstackentry{#1}{lasty}-\sseq@y\relax
    }{
        \the\numexpr\sseq@getstackentry{0}{lasty}-\sseq@y\relax#1 % #1 isn't a number so put it back
    }
}

\def\sseq@lastclass#1{
    \@xp\sseq@ifempty\@xp{\romannumeral-0#1}{
        \sseq@getstackentry{#1}{lastclass}
    }{
        \sseq@getstackentry{0}{lastclass}#1 % #1 isn't a number so put it back
    }
}


\sseq@DeclareDocumentCommand\nameclass {mr()} {
    \bgroup
    \sseq@setthiscall{\sseq@nameclass#1(#2)}
    \let\sseq@possibleoninputlinepar\empty
    \sseq@parsecoord\coord{#2}{}
    \sseq@cleanup@obj{class.namedclass.#1}
    \sseq@obj@xdef{class.namedclass.#1}{\coordnopar}
    \sseq@breakpoint
    \egroup
}

\sseq@DeclareDocumentCommand\tagclass{mr()}{
    \bgroup
    \sseq@setthiscall{\tagclass#1(#2)}
    \let\sseq@possibleoninputlinepar\empty
    \def\sseq@class@tag{#1}
    \sseq@parsecoord\coord{#2}{}%
    \sseq@obj@ifdef{partcoord.(\sseq@xycoord).tag.\sseq@class@tagprefix\sseq@class@tag}{%
        \sseq@error@nn{class-tag-already-defined}{\sseq@class@tagprefix\sseq@class@tag}{(\sseq@xycoord)}% could be warning
    }%
    \sseq@cleanup@obj{class.\coord.tag}%
    \sseq@obj@xdef{class.\coord.tag}{\sseq@class@tagprefix\sseq@class@tag}%
    \sseq@cleanup@obj{partcoord.(\sseq@xycoord).tag.\sseq@class@tagprefix\sseq@class@tag}%
    \sseq@obj@xdef{partcoord.(\sseq@xycoord).tag.\sseq@class@tagprefix\sseq@class@tag}{\sseq@index}%
    \sseq@breakpoint
    \egroup
}

\sseq@DeclareDocumentCommand\gettag{mr()}{
    \bgroup
    \sseq@setthiscall{\gettag#1(#2)}
    \let\sseq@possibleoninputlinepar\empty
    \sseq@parsecoord\coord{#2}{}
    \sseq@obj@ifdef{class.\coord.tag}{
        \xdef#1{\sseq@obj{class.\coord.tag}}
    }{
        \sseq@error@n{class-no-tag}{(#2)}
    }
    \sseq@breakpoint
    \egroup
}

\prg_new_conditional:Npnn \sseq_if_out_of_bounds:n #1 { T ,  F , TF } {
    \bgroup
    \sseq@tempiftrue
    \sseq@parsecoord\coord{#1}{}
    \@gobbletwo\sseq@breakpoint\sseq@tempiffalse % this executes \sseq@tempiffalse if we broke, otherwise it gets gobbled
    \ifsseq@tempif
        \sseq_if_out_of_bounds_noparse:nnTF { \xcoord } { \ycoord }{\egroup\prg_return_true:}{\egroup\prg_return_false:}
    \else
        \egroup
        \prg_return_false
    \fi
}

% For internal use b/c it's faster not to call \sseq@parsecoord
% There's no noticeable performance boost though... =(
\prg_new_conditional:Npnn \sseq_if_out_of_bounds_noparse:nn #1#2 {TF} {
    \sseq@tempiftrue
    \ifnum#1<\sseq@xmaxpp\relax\ifnum#1>\sseq@xminmm\relax\ifnum#2<\sseq@ymaxpp\relax\ifnum#2>\sseq@yminmm\relax
        \sseq@tempiffalse
    \fi\fi\fi\fi
    \ifsseq@tempif
        \prg_return_true:
    \else
        \prg_return_false:
    \fi
}


\sseq@DeclareDocumentCommand \IfOutOfBoundsTF { r() } { \sseq_if_out_of_bounds:nTF { #1 } }
\sseq@DeclareDocumentCommand \IfOutOfBoundsT  { r() } { \sseq_if_out_of_bounds:nT  { #1 } }
\sseq@DeclareDocumentCommand \IfOutOfBoundsF  { r() } { \sseq_if_out_of_bounds:nF  { #1 } }
\sseq@DeclareDocumentCommand \IfInBoundsTF { r() mm } { \sseq_if_out_of_bounds:nTF { #1 } { #3 } { #2 } }
\sseq@DeclareDocumentCommand \IfInBoundsT  { r() } { \sseq_if_out_of_bounds:nF  { #1 } }
\sseq@DeclareDocumentCommand \IfInBoundsF  { r() } { \sseq_if_out_of_bounds:nT  { #1 } }


%%%%%%
%%%
%%%   Tikz Primitives
%%%
%%%%%%

% Replace a tikz command with a command that saves the command on savedpaths

\def\sseq@modtikzcommands{%
    \let\scope\sseq@scope
    \let\endscope\sseq@endscope
    \@xp\sseq@modtikzcommands@\sseq@tikzcommands\sseq@nil
}
\def\sseq@modtikzcommands@#1{%
    \ifx#1\sseq@nil\else
        \protected\edef#1{\@xp\@nx\csname\sseq@macroname#1\space\endcsname\@nx#1}%
        \@xp\let\csname\sseq@macroname#1\space\endcsname\sseq@defer@tikzcommand % This is just for error reporting so that it will say "Paragraph ended before \draw  was complete"
        \@xp\sseq@modtikzcommands@
    \fi
}

\def\sseq@defer@tikzcommand#1{%
    \begingroup
    \let\sseq@isaliveprotect\sseq@isaliveprotect@protect
    \sseq@setinputline
    \sseq@loadinputline
    \sseq@callas{#1}%
    \gdef\sseq@thepathsofar{#1[/utils/exec={\sseq@thesseqstyle\sseq@thetikzprimitivestyle\the\sseq@scope@toks\sseq@savedoptioncode}]}%
    \xdef\sseq@whattheusersaid{\string#1}
    \sseq@defer@tikzcommand@
}

\def\sseq@defer@tikzcommand@{%
    \futurelet\testtok\sseq@defer@tikzcommand@@
}

\def\sseq@defer@tikzcommand@@{%
    \ifx\testtok;
        \let\next\sseq@defer@tikzcommand@finish
    \else
        \ifx\testtok[%
            \let\next\sseq@defer@tikzcommand@option
        \else
            \ifx\testtok(%
                \let\next\sseq@defer@tikzcommand@coord
            \else
                \@xp\ifx\space\testtok
                    \let\next\sseq@defer@tikzcommand@space
                \else
                    \ifx\testtok\bgroup
                        \let\next\sseq@defer@tikzcommand@group
                    \else
                        \let\next\sseq@defer@tikzcommand@other
                    \fi
                \fi
            \fi
        \fi
    \fi
    \sseq@call{\next}%
}


\def\sseq@defer@tikzcommand@option[#1]{%
    \sseq@processoptions{tikz ~ primitives}{#1}%
    \ifx\sseq@savedoptioncode\pgfutil@empty\else
        \sseq@x@addto@macro\sseq@thepathsofar{[/utils/exec={\unexpanded\@xp{\@xp\sseq@options@bothpassmode\sseq@savedoptioncode}}]}%
    \fi
    \sseq@g@addto@macro\sseq@whattheusersaid{[#1]}
    \sseq@defer@tikzcommand@
}

\def\sseq@defer@tikzcommand@coord(#1){
    \sseq@tikzprimitives@coord(#1)
    \sseq@g@addto@macro\sseq@whattheusersaid{(#1)}
    \sseq@defer@tikzcommand@
}

% When do groups occur in tikz commands? should we try to parse the inside?
\def\sseq@defer@tikzcommand@group#1{
    \sseq@g@addto@macro\sseq@whattheusersaid{{#1}}
    \sseq@xprotected@addto@macro\sseq@thepathsofar{{#1}}%
    \sseq@defer@tikzcommand@
}

\long\def\sseq@defer@tikzcommand@other#1{
    \ifcat$\@xp\@gobble\string#1$ % test for control word. Won't catch active characters.
        \@xp\use_i:nnn
    \else
        \@xp\ifx\csname sseq@illegalintikz@\string#1\endcsname\relax
            \exp_last_unbraced:Nf \use_ii:nnn
        \else
            \exp_last_unbraced:Nf \use_iii:nnn
        \fi
    \fi{  % case i: it's not a control word. Just add it to the saved path
        \sseq@g@addto@macro\sseq@whattheusersaid{#1}
        \sseq@g@addto@macro\sseq@thepathsofar{#1}%
        \sseq@defer@tikzcommand@
    }{ % case ii: it's some different control word. Let's try full expanding it.
       % note that here we cannot figure out what the user actually said, so we just have to report the f-expanded version.
        \def\tempa{#1} % save the current value
        \exp_last_unbraced:Nf \sseq@defer@tikzcommand@other@cs#1
    }{ % case iii: it's an illegal control word (e.g., \class, \d, \begin, \end, ...) throw an error.
        \sseq@error@xx{incomplete-tikz}{\unexpanded\@xp{\sseq@whattheusersaid}}{
            \ifx\par#1 the ~ start ~ of ~ a ~ new ~ paragraph
            \else \string#1\fi
        }
        \sseq@defer@tikzcommand@finish@ % finish@ doesn't try to grab a semicolon, which is good because there isn't one.
        #1 % Reinsert the token that is causing us to stop
    }
}

\def\sseq@defer@tikzcommand@other@cs#1{ % so we f expanded the control sequence, now we test if that did anything
    \def\tempb{#1} % get the first f expanded token
    \ifx\tempa\tempb
        \@xp\@firstoftwo
    \else
        \@xp\@secondoftwo
    \fi{% it didn't get expanded any more, so just add it to the stored stuff
        \sseq@g@addto@macro\sseq@whattheusersaid{#1}
        \sseq@g@addto@macro\sseq@thepathsofar{#1}
        \sseq@defer@tikzcommand@
    }{% it got expanded. Run it through the main loop again.
        \sseq@defer@tikzcommand@#1
    }
}

% Illegal control words:
\long\def\sseq@setillegalcontrolwords#1{\sseq@setillegalcontrolwords@#1\sseq@nil}
\long\def\sseq@setillegalcontrolwords@#1{
    \ifx#1\sseq@nil\else
        \@xp\def\csname sseq@illegalintikz@\string#1\endcsname{}
        \@xp\sseq@setillegalcontrolwords@
    \fi
}

% We missing anything here?
\sseq@setillegalcontrolwords{
    \class\classoptions\replaceclass\d\doptions\structline\structlineoptions\circleclasses
    \savestack\restorestack\pushstack\nameclass\foreach\begin\end\par
    \clip\coordinate\draw\fill\filldraw
    \graph\matrix\node\path\pattern
    \shade\shadedraw\useasboundingbox
}


\@xp\def\@xp\sseq@defer@tikzcommand@space\space{%
    \sseq@g@addto@macro\sseq@thepathsofar{~}%
    \sseq@g@addto@macro\sseq@whattheusersaid{~}
    \sseq@defer@tikzcommand@
}

\def\sseq@defer@tikzcommand@finish;{
    \sseq@g@addto@macro\sseq@whattheusersaid{;}
    \sseq@defer@tikzcommand@finish@
}

% If the expression was incomplete, we'll jump here to avoid adding the semicolon to the error printout
\def\sseq@defer@tikzcommand@finish@{
    \sseq@g@addto@macro\sseq@thepathsofar{;}
    \global\sseq@thiscalltoks\@xp{\sseq@whattheusersaid}
    \ifx\sseq@pageconstraint\sseq@pageconstraint@true
        \sseq@savedpaths@xaddtikzpath{\unexpanded\@xp{\sseq@thepathsofar}}%
    \else
        \sseq@savedpaths@xaddtikzpath{%
            \@nx\sseq@tikzcommand@conditionaldraw{\unexpanded\@xp{\sseq@thepathsofar}}{\unexpanded\@xp{\sseq@pageconstraint}}
        }%
    \fi
    \endgroup
}


\def\sseq@tikzcommand@conditionaldraw#1#2{%
    \sseq@pgfmathparse@rescan{#2}%
    \ifnum\pgfmathresult>\z@ #1 \fi
}

\def\sseq@pgfmathparse@rescan#1{\makeatletter\catcode`\&=12\relax\scantokens{\pgfmathparse{#1}}}

%%% Coordinate parser, copied with huge simplifications from \tikz@@scan@@no@calculator, tikz.code.tex line 4994.
%%% We don't need to handle any of the weird cases, we just need to know about them so we can give up and let tikz do the work later.
% TODO: do same coordinate fixing for coordinates in calculations (yeah right! not even clear this is a good idea...)
\def\sseq@tikzprimitives@coord(#1){
    \begingroup
    \let\next\sseq@tikzprimitives@coords@maybenamedclass
    \ifsseq@tikzprims@integershift
        \pgfutil@in@${#1}
        \ifpgfutil@in@
            \let\next\sseq@tikzprimitives@coords@handlemath
        \else
            \pgfutil@in@ :{#1}
            \ifpgfutil@in@\else
                \pgfutil@in@{intersection }{#1}
                \ifpgfutil@in@\else
                    \pgfutil@in@|{#1}%
                    \ifpgfutil@in@\else
                        \pgfutil@in@,{#1}
                        \ifpgfutil@in@
                            \let\next\sseq@tikzprimitives@coords@maybeclass
                        \fi
                    \fi
                \fi
            \fi
        \fi
    \fi
    \next{#1}
    \endgroup
}
\def\sseq@tikzprimitives@coords@maybeclass#1{
    \let\next\sseq@tikzprimitives@coords@notaclass % most branches do this.
    % First check whether there are any of the protect "to be determined later" variables like \xmax, etc. If there are, it's not a class
    \sseq@protected@edef\sseq@temp@i{#1}
    \edef\sseq@temp@ii{#1}
    \ifx\sseq@temp@i\sseq@temp@ii
        \@xp\sseq@tikzprimitive@getcoord@anchor#1.\sseq@nil % puts coord into \sseq@tempcoord, anchor if any into \sseq@tempanchor
        \exp_args:NNo\pgfutil@in@.{\sseq@tempanchornopt} % If there's an extra . in the "anchor", it's not an anchor, and this is not a class
        \ifpgfutil@in@\else
            \exp_args:NNo \pgfutil@in@,{\sseq@tempcoord} % If the . we used to delineate the "anchor" was in the x-coordinate, it's not an anchor and this is not a class
            \ifpgfutil@in@
                \exp_args:NNo \pgfutil@in@,{\sseq@tempanchornopt} % If there is a comma in the "anchor" it's not an anchor and this is not a class
                \ifpgfutil@in@\else
                    \ifx\sseq@tempanchor\pgfutil@empty
                        \sseq@tempiftrue
                    \else % If there's an "anchor" we need to test whether the expression is a valid decimal coordinate. If it is, we're going to treat it as not a class
                        % outputs into sseq@tempif, sets sseq@tempiffalse if it IS valid math, b/c then we're not a anchor
                        \sseq@anchortrue % If this ends up getting handled as a tikz coordinate, we'll have to add an "anchor ignored" error
                        \sseq@tikzprimitives@testlastcoord@validmathexpression#1,\sseq@nil
                    \fi
                    \ifsseq@tempif
                        \@xp\sseq@tikzprimitives@ifintcoords\@xp{\sseq@tempcoord}{ % check we're all integer coordinates
                            \sseq@parsecoord\coord{\sseq@tempcoord}{tikz ~ primitive}%
                            \ifx\sseq@index\pgfutil@empty\else
                                \sseq@x@addto@macro\sseq@thepathsofar{(sseq{\sseq@removeparens\coord}\sseq@tempanchor)}% Okay, we're all set, it's a class
                                \let\next\@gobble % don't run \sseq@tikzprimitives@notaclass
                            \fi
                        }{}%
                    \fi
                \fi
            \fi
        \fi
    \fi
    \next{#1}
}

% Sets tempiffalse if if the second coordinate IS a valid math expression -- then t
\def\sseq@tikzprimitives@testlastcoord@validmathexpression#1,#2,#3\sseq@nil{
    \sseq@ifpgfmathexpr{#2}{
        \sseq@tempiffalse
    }{
        \sseq@tempiftrue
    }
}

\def\sseq@tikzprimitives@coords@maybenamedclass#1{
    \sseq@tikzprimitive@getcoord@anchor#1.\sseq@nil % puts coord into \sseq@tempcoord, anchor if any into \sseq@tempanchor
    \sseq@protectedeval{\@nx\sseq@obj@ifdef{class.namedclass.\@nx\detokenize{\sseq@tempcoord}}}{% need this detokenize here to prevent it from throwing errors when there's a command inside...
        \sseq@x@addto@macro\sseq@thepathsofar{(sseq{\sseq@obj{class.namedclass.\sseq@tempcoord}\sseq@tempanchor})}
    }{
        \sseq@tikzprimitives@coords@notaclass{#1}
    }
}

\let\sseq@tikzprimitives@coords@maybeclass@save\sseq@tikzprimitives@coords@maybeclass


\def\sseq@tikzprimitives@coords@notaclass#1{
    \def\next{\sseq@tikzprimitives@coords@notaclass@leavetotikz{#1}}
    \pgfutil@in@${#1}
    \ifpgfutil@in@\else
        \pgfutil@in@ :{#1}
        \ifpgfutil@in@\else
            \pgfutil@in@{intersection }{#1}
            \ifpgfutil@in@\else
                \pgfutil@in@|{#1}%
                \ifpgfutil@in@\else
                    \pgfutil@in@,{#1}
                    \ifpgfutil@in@
                        \def\next{\sseq@tikzprimitives@coords@notaclass@handle{#1}}
                    \fi
                \fi
            \fi
        \fi
    \fi
    \next
}

\def\sseq@tikzprimitives@coords@notaclass@leavetotikz#1{
    \sseq@g@addto@macro\sseq@thepathsofar{(#1)}
}

\def\sseq@tikzprimitives@coords@notaclass@handle#1{
    \ifsseq@anchor
        % Still seems impossible to trigger this?
        \sseq@error@xxx{anchor-ignored}{(#1)}{(\unexpanded\@xp{\sseq@tempcoord})}{\sseq@tempanchor}
        \@xp\sseq@tikzprimitives@coords@notaclass@handle@\sseq@tempcoord,\sseq@nil
    \else
        \sseq@tikzprimitives@coords@notaclass@handle@#1,\sseq@nil
    \fi
}

\def\sseq@tikzprimitives@coords@notaclass@handle@#1,#2,#3\sseq@nil{
    \sseq@ifempty{#3}{
        \let\sseq@index\pgfutil@empty
    }{
        \edef\sseq@index{,\unexpanded\@xp{\sseq@removecomma#3\sseq@nil}}
        \edef\sseq@indexnocomma{\unexpanded\@xp{\sseq@removecomma#3\sseq@nil}}
    }
    \sseq@ifpgfmathexpr{#1}{
        % Decide whether we can keep the result, which is stored in \sseq@mathresult
        \protected@edef\sseq@tempa{#1}
        \edef\sseq@tempb{#1}
        \ifx\sseq@tempa\sseq@tempb
            \let\sseq@tempx\sseq@mathresult
        \else
            \def\sseq@tempx{#1}
        \fi
    }{
        \sseq@error@xx{invalid-tikz-coord}{(#1,#2\sseq@index)}{x}
        \def\sseq@defer@tikzcommand@finish;{\endgroup}
    }
    \sseq@ifpgfmathexpr{#2}{
        % Decide whether we can keep the result, which is stored in \sseq@mathresult
        \protected@edef\sseq@tempa{#2}
        \edef\sseq@tempb{#2}
        \ifx\sseq@tempa\sseq@tempb
            \let\sseq@tempy\sseq@mathresult
        \else
            \def\sseq@tempy{#2}
        \fi
    }{
        \sseq@error@xx{invalid-tikz-coord}{(#1,#2\sseq@index)}{y}
        \def\sseq@defer@tikzcommand@finish;{\endgroup}
    }
    \def\sseq@tempa{#1}
    \ifx\sseq@index\pgfutil@empty\else
        \sseq@error@xxx{index-ignored}{(\unexpanded{#1,#2}\unexpanded\@xp{\sseq@index})}{\unexpanded{(#1,#2)}}{\unexpanded\@xp{\sseq@indexnocomma}}
    \fi
    \sseq@xprotected@addto@macro\sseq@thepathsofar{(\sseq@tempx,\sseq@tempy)}
}

\def\sseq@tikzprimitives@ifintcoords#1{\sseq@tikzprimitives@ifintcoords@#1,\sseq@nil}
\def\sseq@tikzprimitives@ifintcoords@#1,#2,#3\sseq@nil{%
    \sseq@ifintexpr{#1}{%
        \sseq@ifintexpr{#2}{%
            \@firstoftwo
        }{\@secondoftwo}%
    }{\@secondoftwo}%
}
\def\sseq@tikzprimitives@coords@handlemath#1{\sseq@tikzprimitives@coords@handlemath@(#1)}
\def\sseq@tikzprimitives@coords@handlemath@#1$#2$){\sseq@g@addto@macro\sseq@thepathsofar{#1$#2$)}}
\def\sseq@tikzprimitive@getcoord@anchor#1.#2\sseq@nil{
    \def\sseq@tempcoord{#1}
    \ifx\sseq@nil#2\sseq@nil
        \def\sseq@tempanchor{}
        \def\sseq@tempanchornopt{}
    \else
        \edef\sseq@tempanchor{.\sseq@tikzprimitive@getcoord@anchor@eatdot#2}
        \edef\sseq@tempanchornopt{\sseq@tikzprimitive@getcoord@anchor@eatdot#2}
    \fi
}
\def\sseq@tikzprimitive@getcoord@anchor@eatdot#1.{#1}
\def\sseq@uptopt#1.#2\sseq@nil{#1}
%%% Page constraint and \isalive
% #1 -- the new page constraint
% #2 -- a binary logical operator (&& or ||).
\def\sseq@updatepageconstraint#1#2{
    \def\sseq@isaliveprotect{}
    \let\sseq@isalive@\sseq@isalive@active
    \bgroup
    % Was pretty hard to get \isalive to report errors correctly. And to work in general >_<
    \let\sseq@isalive@parens\sseq@isalive@parens@check
    \sseq@d@addto@macro\sseq@error@hook{\let\protect\relax}%
    \let\protect\@unexpandable@protect
    \edef\sseq@error@annotation{\unexpanded{.^^J(In page constraint "#1")}\unexpanded\@xp{\sseq@error@annotation}}
    \selectfont\sseq@restorefont
    \setbox0=\hbox{#1}
    \egroup
    \sseq@protected@edef\sseq@pageconstraint{(\unexpanded\@xp{\sseq@pageconstraint})#2(#1)}
}


\def\sseq@updatepageconstraintrange{%
    \sseq@protected@edef\sseq@pageconstraint{%
        (\unexpanded\@xp{\sseq@pageconstraint})&&(\temp<=\page \ifx\tempmax\empty\else&&\page<=\tempmax\fi)%
    }%
}

% This protect variant allows us to delay evaluation of \isalive, just so that we can
% capture something close to the original argument to (does this matter anymore?)
\let\sseq@isaliveprotect\relax
\def\sseq@isaliveprotect@protect{\@nx\sseq@isaliveprotect\@nx}
\def\sseq@isalive{\sseq@isaliveprotect\sseq@isalive@}
\def\sseq@isalive@{\sseq@isalive@error}
\def\sseq@isalive@error{\sseq@error{is-alive-illegal-here}}
\def\sseq@isalive@active#1{%
    \@xp\ifx\@xp$\@gobble#1$%
        \@xp\sseq@isalive@parens\@xp#1%
    \else
        \sseq@isalive@list#1\sseq@nil
    \fi
}
\def\sseq@isalive@list(#1)#2{%
    \sseq@isalive@parens(#1)%
    \ifx\sseq@nil#2\else
        &&\@xp\sseq@isalive@list\@xp#2%
    \fi
}

\def\sseq@isalive@parens@check(#1){
    \sseq@parsecoord\coord{#1}{\string\isalive}
}
\def\sseq@isalive@parens(#1){\sseq@parsecoordex\sseq@isalive@parens@(#1)}
\def\sseq@isalive@parens@(#1,#2,#3){\@nx\sseq@isalive@final{#1}{#2}{#3}{\sseq@obj{class.(#1,#2,#3).num}}}
\def\sseq@isalive@final#1#2#3#4{%
    \ifnum\sseq@obj{class.(#1,#2,#3)[#4].page}<\sseq@thepagecount
        0%
    \else
        \ifnum#4=\z@
            1%
        \else
            \ifnum\sseq@obj{class.(#1,#2,#3)[\the\numexpr #4-1\relax].page}=\sseq@thepagecount
                0%
            \else
                1%
            \fi
        \fi
    \fi
}





%%%
%%% This is a macro for typesetting monomials.
%%%

% We need to check for math subscript characters
\char_set_catcode_math_subscript:N \_

\protected\def\sseqnormalizemonomial#1{
    \bgroup
    \def\sseq@support{}
    \sseq@ifempty{#1}{1}{
        \sseqnormalizemonomial@#1\sseq@nil
    }
}
\def\sseqnormalizemonomial@#1{
    \pgfkeys@spdef\sseq@var{#1}
    \def\sseq@power{1}
    \def\sseq@subscript{}
    \futurelet\testtok\sseqnormalizemonomial@@
}

\def\sseqnormalizemonomial@@{
    \ifx\testtok^
        \let\next\sseqnormalizemonomial@sup
    \else
        \ifx\testtok_
            \let\next\sseqnormalizemonomial@sub
        \else
            \let\next\sseqnormalizemonomial@add
        \fi
    \fi
    \next
}

\def\sseqnormalizemonomial@sup^#1{
    \edef\sseq@power{\the\numexpr#1\relax}
    \futurelet\testtok\sseqnormalizemonomial@@
}

\def\sseqnormalizemonomial@sub_#1{
    \def\sseq@subscript{_{#1}}
    \futurelet\testtok\sseqnormalizemonomial@@
}

\def\sseqnormalizemonomial@add{
    \def\temp{1}
    \ifx\sseq@var\temp\else
        \edef\sseq@var{\unexpanded\@xp{\sseq@var}\unexpanded\@xp{\sseq@subscript}}
        \@ifundefined{\sseqnormalizemonomial@varcs}{
            \@xp\edef\csname\sseqnormalizemonomial@varcs\endcsname{\sseq@power}
            \sseq@e@addto@macro\sseq@support{\@nx\\{\unexpanded\@xp{\sseq@var}}{\sseqnormalizemonomial@varcs}}%
        }{
            \@xp\edef\csname\sseqnormalizemonomial@varcs\endcsname{\the\numexpr\csname\sseqnormalizemonomial@varcs\endcsname + \sseq@power}
        }
    \fi
    \sseqnormalizemonomial@testend
}


\def\sseqnormalizemonomial@testend{
    \ifx\testtok\sseq@nil
        \sseqnormalizemonomial@done
        \sseq@smuggle@macro\processedlabel
        \egroup
        \let\next\@gobble
    \else
        \@xp\ifx\space\testtok
            \def\next{\@xp\futurelet\@xp\testtok\@xp\sseqnormalizemonomial@testend\romannumeral-`0}
        \else
            \let\next\sseqnormalizemonomial@
        \fi
    \fi
    \next
}

\def\sseqnormalizemonomial@varcs{sseqnormalizemonomial@var@\detokenize\@xp{\sseq@var}}

\def\sseqnormalizemonomial@done{
    \def\processedlabel{}
    \def\\##1##2{
        \sseq@tempcount=\csname ##2\endcsname\relax
        \ifnum\sseq@tempcount=\z@
        \else
            \ifnum\sseq@tempcount=\@ne
                \sseq@d@addto@macro\processedlabel{##1}
            \else
                \sseq@e@addto@macro\processedlabel{\unexpanded{##1}^{\the\sseq@tempcount}}
            \fi
        \fi
    }
    \sseq@support
    \ifx\processedlabel\pgfutil@empty % use 1 for empty monomial
        \def\processedlabel{1}
    \fi
}

% Restore catcode of underscore to letter
\catcode`\_ = 11\relax




\ExplSyntaxOff 