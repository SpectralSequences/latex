%%
%% Description: The May spectral sequence for $Ext(\HF_{2*}bo)$
%%
%%    This is the May spectral sequence computing Ext_{A(1)} ( F_2, F_2 ), which by a change of rings and a simple argument that the ASS has to collapse,
%%    is telling us pi_* bo. See the Green Book for reference.
%%    The main purpose of this example from my perspective is to illustrate the utility of class "tags".
%%


\documentclass[tooltips]{spectralsequence-example}

\NewSseqCommand\tower {} {
    \class(0,0)
    \foreach\i in {1,...,11}{
        \class(0,\i)
        \structline(0,\i-1,-1)(0,\i,-1)
    }
}
\NewSseqGroup\hvee {} {
    \tower
    \foreach\i in {1,...,11}{
        \class(\i,\i)
        \structline(\i-1,\i-1,-1)(\i,\i,-1)
    }
}

\begin{document}
\begin{sseqdata}[
    name=A1 May,
    x range={0}{10}, y range={0}{10},
    degree={-1}{1},
    classes={fill, tooltip = {(\xcoord,\ycoord)}},
    class labels={below=3pt},
    run off differentials= ->,
    no orphan edges,
]
\foreach\i in {0,1,2}{
    \begin{scope}[xshift=4*\i,yshift=2*\i,tag prefix=\i.]
    \hvee[tag=id](0,0)
    \hvee[tag=h0h1](1,2)
    \hvee[tag=h20](2,1)
    \foreach\j in {1,...,10}{
        \d1(2,\j,h20,h0h1)
    }
    \foreach\j in {1,...,10}{
        \d1(2+\j,1+\j,h20,h0h1)
    }
    \end{scope}
}


\foreach\i in {0,...,7}{
    \d2(4+\i,2+\i,1.id,0.id)
}
\class(12,6)
\d2(12,6,,1)

%\classoptions["h_0"](0,1)
\classoptions["h_1"](1,1)
\classoptions["h_{20}"](2,1)
\classoptions["h_{20}^2"](4,2)
\classoptions["h_{20}^3"](6,3)
\classoptions["h_{20}^4"](8,4)
\classoptions["h_{20}^5"](10,5)

\doptions[shorten >=3pt]1(10,10,2.h20,2.h0h1)
\end{sseqdata}

\printpage[name=A1 May,page=1]
\newpage
\printpage[name=A1 May,page=2]
\newpage
\begin{sseqpage}[name=A1 May,page=3]
\classoptions["h_0h_{20}^2"](4,3,1.id)
\end{sseqpage}
\end{document} 