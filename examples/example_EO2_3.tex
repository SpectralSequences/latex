%%
%% Package: spectralsequences.sty version 1.0.0
%% Author: Hood Chatham
%% Email: hood@mit.edu
%% Date: 2017-06-18
%% License: Latex Project Public License
%%
%% File: example_EO2_3.tex
%%
%%    This is the homotopy fixed point spectral sequence for EO_2 at the prime 3. The maximal finite subgroup of the Morava stabilizer for E_{p-1} is
%%    of size 2p(p-1)^2 = 24, and so there is a norm element v in degree 24. There's also a bunch of trace classes on the zero line, but the trace map
%%    E_* --> H^*( G ; E_* ) is induced by the trace map E_n --> EO_n, so all of these classes are permanent cycles. They are hard to compute and we don't draw them.
%%    We also have classes \alpha and \beta coming from the stabilizer action, which are the images of \alpha_1 and \beta_1 in the ANSS.
%%    By looking at cobar representatives, we can see that v*\beta_1 is the image of \beta_{3/3}.
%%    Thus, the Toda differential in the ANSS d_3(\beta_{3/3}) = \alpha \beta^3 forces also that d_3(v) = \alpha \beta^2. Likewise, the Toda "Kudo" differential
%%    d_9( \alpha \beta_{3/3}^2 ) = \beta^7 gives us upon dividing by \beta twice that d_9(\alpha v^2) = \beta^5. At this point, there are no possible differentials.
%%    We see that v^3 survives so EO_n* is 72 = 2p^2(p-1)^2 periodic.  The picture is exactly the same at other odd primes. At 2, this degenerates to the
%%    HFPSS for KO = KU^{hC_2} (see example_KUHFPSS).
%%

\documentclass{article}
\usepackage{spectralsequences}
\usepackage[landscape,margin=0cm,top=2cm]{geometry}

\begin{document}
\begin{sseqdata}[name=EO(2),Adams grading,
    y range={0}{14},x range={0}{160},
    xscale=0.15, x tick step=5,
    classes={fill, tooltip={(\xcoord,\ycoord)}}
]
\foreach \v in {0,...,8}{
    \foreach \b in {0,...,11}{
        \foreach \a in {0,1}{
            \class(3*\a + 10*\b+24*\v,\a+2*\b)
            \ifnum\b>0\relax
                \structline(3*\a+10*\b-10+24*\v,\a+2*\b-2) (3*\a + 10*\b+24*\v,\a+2*\b)
            \fi
        }
        \structline(10*\b+24*\v,2*\b)(3 + 10*\b + 24*\v,2*\b+1)
        \ifnum \v = \numexpr\v/3*3\relax

        \else
            \ifnum\b<9\relax
                \d5(10*\b+24*\v,2*\b)
            \fi
        \fi
    }
}

% v^2ab^2 is in degree 2*24 + 3 + 2*10 = 71, 5
% b^{pn+1} = b^{7} is in degree 7*10 = 70,14
\foreach \v in {2,5}{
    \foreach \b in {0,...,6}{
        \d9(\v*24 + 3 +10*\b,1+2*\b)
    }
}
\end{sseqdata}
\printpage[name=EO(2),page=0]
\newpage
\printpage[name=EO(2),page=5]
\newpage
\printpage[name=EO(2),page=9]
\newpage
\begin{sseqpage}[name=EO(2),page=10]
\classoptions["a" left](3,1)
\classoptions["b" right](10,2)
\classoptions["ab" left](13,3)
\classoptions["b^2" right](20,4)
\classoptions["b^3" right](30,6)
\classoptions["b^4" right](40,8)

\classoptions["v^3" right](72,0)
\end{sseqpage}
\end{document} 