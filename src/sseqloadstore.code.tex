%%
%% Package: spectralsequences v1.2.2
%% Author: Hood Chatham
%% Email: hood@mit.edu
%% Date: 2021-07-18
%% License: Latex Project Public License
%%
%% File: sseqloadstore.code.tex
%%
%%    Defines commands for maintaining separate namespaces between distinct spectral sequences and saving information.
%%


% These are attributes that come with a given spectral sequence (by name)
% All of these values are loaded at the beginning of sseqpage (and of sseqdata if an existing sseq is being modified).
% They are then stored together at the end of sseqdata (and of sseqpage if "keep changes" option is present).
\def\sseq@storelist{}
\def\sseq@storelist@setdefaults{}
\def\sseq@addtostorelist#1#2{%
    \sseq@g@addto@macro\sseq@storelist{\\#1}%
    \sseq@x@addto@macro\sseq@storelist@setdefaults{\global\let\@nx#1\@xp\@nx\csname \sseq@macroname#1@default\endcsname}%
    \@xp\gdef\csname\sseq@macroname#1@default\endcsname{#2}%
}
% TODO: is this used?
\def\sseq@addtostorelistlet#1#2{%
    \sseq@g@addto@macro\sseq@storelist{\\#1}%
    \sseq@x@addto@macro\sseq@storelist@setdefaults{\global\let\@nx#1\@xp\@nx\csname \sseq@macroname#1@default\endcsname}%
    \global\@xp\let\csname\sseq@macroname#1@default\endcsname#2%
}

\def\sseq@addiftostorelist#1{\sseq@addiftostorelistopt{#1}{false}}
\def\sseq@addiftostorelist@truedefault#1{\sseq@addiftostorelistopt{#1}{true}}
\def\sseq@addiftostorelistopt#1#2{%
    \@xp\newif\csname ifsseq@#1\endcsname
    \sseq@g@addto@macro\sseq@storelist{\sseq@storelistif{#1}}%
    \sseq@x@addto@macro\sseq@storelist@setdefaults{\@xp\@nx\csname sseq@#1@default\endcsname}%
    \@xp\xdef\csname sseq@#1@default\endcsname{\global\@xp\@nx\csname sseq@#1#2\endcsname}%
}
\def\sseq@addkeytostorelist#1#2{
    \sseq@g@addto@macro\sseq@storelist{\sseq@storelistkey{#1}}%
    \sseq@x@addto@macro\sseq@storelist@setdefaults{
        \global\let\@xp\@nx\csname pgfk@#1/.@cmd\endcsname\@xp\@nx\csname pgfk@#1/.@cmd@default\endcsname
        \global\let\@xp\@nx\csname pgfk@#1/.@body\endcsname\@xp\@nx\csname pgfk@#1/.@body@default\endcsname
    }%
    \pgfkeysdef{#1}{#2}%
    \global\@xp\let\csname pgfk@#1/.@cmd@default\@xp\endcsname\csname pgfk@#1/.@cmd\endcsname
    \global\@xp\let\csname pgfk@#1/.@body@default\@xp\endcsname\csname pgfk@#1/.@body\endcsname
}

% transfer data about the current spectral sequence between \<command> and \<command><sseqname>
\def\sseq@storelist@store#1{\global\@xp\let\csname\sseq@macroname#1@@@\sseq@thename\endcsname#1}
\def\sseq@storelist@get#1{\global\@xp\let\@xp#1\csname\sseq@macroname#1@@@\sseq@thename\endcsname}

\def\sseq@storelistif@store#1{%
    \@xp\edef\csname sseq@storeif@#1\endcsname{\csname ifsseq@#1\endcsname \@xp\@nx\csname sseq@#1true\endcsname\else\@xp\@nx\csname sseq@#1false\endcsname\fi}%
    \@xp\sseq@storelist@store\csname sseq@storeif@#1\endcsname
}
\def\sseq@storelistif@get#1{
    \@xp\sseq@storelist@get\csname sseq@storeif@#1\endcsname
    \csname sseq@storeif@#1\endcsname
}

\def\sseq@storelistkey@store#1{
    \global\@xp\let\csname pgfk@#1/.@cmd@@@\sseq@thename\@xp\endcsname\csname pgfk@#1/.@cmd\endcsname
    \global\@xp\let\csname pgfk@#1/.@body@@@\sseq@thename\@xp\endcsname\csname pgfk@#1/.@body\endcsname
}
\def\sseq@storelistkey@get#1{
    \global\@xp\let\csname pgfk@#1/.@cmd\@xp\endcsname\csname pgfk@#1/.@cmd@@@\sseq@thename\endcsname
    \global\@xp\let\csname pgfk@#1/.@body\@xp\endcsname\csname pgfk@#1/.@body@@@\sseq@thename\endcsname
}

\def\sseq@storecmds{\let\\\sseq@storelist@store\let\sseq@storelistif\sseq@storelistif@store\let\sseq@storelistkey\sseq@storelistkey@store\sseq@storelist}
\def\sseq@getcmds{\let\\\sseq@storelist@get\let\sseq@storelistif\sseq@storelistif@get\let\sseq@storelistkey\sseq@storelistkey@get\sseq@storelist}

%%%% New "poor man's OOP" Load/Store
% Directly stores things into commands that are annotated with the name of the spectral sequence.
% I use this primarily for commands that won't necessarily exist for every spectral sequence
% for instance because it has the description of a particular class or differential as part of its name
\def\sseq@obj@fullname#1{sseq.\sseq@thename.#1}
\def\sseq@obj#1{\csname sseq.\sseq@thename.#1\endcsname}
\def\sseq@obj@gdef#1{\@xptwo\gdef\sseq@obj{#1}}
\def\sseq@obj@xdef#1{\@xptwo\xdef\sseq@obj{#1}}
\def\sseq@obj@protectedxdef#1{\@xptwo\sseq@protected@xdef\sseq@obj{#1}}
\def\sseq@obj@xodef#1{\@xp\gdef\csname \sseq@obj@fullname{#1}\@xp\endcsname\@xp}
\def\sseq@obj@let#1{\@xptwo\let\sseq@obj{#1}} % This shouldn't be used anywhere because all object assignments should be global...
\def\sseq@obj@glet#1{\global\@xptwo\let\sseq@obj{#1}}
\def\sseq@obj@gletcsname#1#2{\global\@xp\let\csname\sseq@obj@fullname{#1}\@xp\endcsname\csname#2\endcsname}
\def\sseq@obj@gletobj#1#2{\global\@xp\let\csname\sseq@obj@fullname{#1}\@xp\endcsname\csname\sseq@obj@fullname{#2}\endcsname}
\def\sseq@obj@undef#1{\global\@xptwo\let\sseq@obj{#1}\relax}
\def\sseq@obj@show#1{\@xptwo\show\sseq@obj{#1}}
\def\sseq@obj@ifundef#1{\@ifundefined{\sseq@obj@fullname{#1}}}
\def\sseq@obj@ifdef#1#2#3{\@ifundefined{\sseq@obj@fullname{#1}}{#3}{#2}}
\def\sseq@obj@useifdef@else#1#2{\@ifundefined{\sseq@obj@fullname{#1}}{#2}{\sseq@obj{#1}}}
\def\sseq@obj@inccheckdef#1{\sseq@obj@ifundef{#1}{\sseq@obj@gdef{#1}{1}}{\sseq@obj@inc{#1}}}
\def\sseq@obj@inc#1{\sseq@obj@xdef{#1}{\the\numexpr\sseq@obj{#1}+1}}
\def\sseq@obj@gaddto#1{\sseq@obj@ifundef{#1}{\sseq@obj@gdef{#1}}{\@xptwo\sseq@g@addto@macro\sseq@obj{#1}}}
\def\sseq@obj@xaddto#1{\sseq@obj@ifundef{#1}{\sseq@obj@xdef{#1}}{\@xptwo\sseq@x@addto@macro\sseq@obj{#1}}}
\def\sseq@obj@xoaddto#1{\sseq@obj@ifundef{#1}{\sseq@obj@xodef{#1}}{\@xp\sseq@g@addto@macro\csname \sseq@obj@fullname{#1}\@xp\endcsname\@xp}}
\def\sseq@obj@map#1#2{\def\\##1{#2}\sseq@obj{#1}}

\def\sseq@lettoobj#1#2{\sseq@obj@ifdef{#2}{\@xp\let\@xp#1\csname\sseq@obj@fullname{#2}\endcsname}{\let#1\empty}}


\def\sseq@obj@ifchanged#1#2#3{\sseq@obj@ifundef{#1.changed}{#3}{#2}}
\def\sseq@obj@changedtrue#1{\sseq@obj@gdef{#1.changed}{}}
\def\sseq@obj@changedfalse#1{\sseq@obj@undef{#1.changed}}

\def\sseq@cleanup@obj#1{
    \ifsseq@keepchanges\else
        \sseq@obj@ifchanged{#1}{}{
            \sseq@obj@changedtrue{#1}
            \sseq@x@addto@macro\sseq@pagecleanup{
                \@xptwo\ifx\sseq@obj{#1}\relax
                    \global\@nx\sseq@obj@let{#1}\relax
                \else
                    \@nx\sseq@obj@gdef{#1}{\unexpanded\@xpthree{\sseq@obj{#1}}}
                \fi
                \@nx\sseq@obj@changedfalse{#1}
            }
        }
    \fi
}


%%%% Saved paths, temporary saved paths, background saved paths
% \sseq@savedpaths@whatever is redefined to be \sseq@savedpaths@whatever@standard or \sseq@savedpaths@whatever@temporary depending on context
% This simplifies the code a lot because I don't have to check again where I should put things.
\sseq@addtostorelist\sseq@savedpaths{}
\def\sseq@tempsavedpaths{}
\def\sseq@savedpaths@addclass@standard#1{\sseq@x@addto@macro\sseq@savedpaths{\@nx\sseq@class@object{#1}}}
\def\sseq@savedpaths@adddifferential@standard#1{\sseq@x@addto@macro\sseq@savedpaths{\@nx\sseq@differential@object{#1}}}
\def\sseq@savedpaths@addextension@standard#1{\sseq@x@addto@macro\sseq@savedpaths{\@nx\sseq@extension@object{#1}}}
\def\sseq@savedpaths@addstructline@standard#1{\sseq@x@addto@macro\sseq@savedpaths{\@nx\sseq@structline@object{#1}}}
\def\sseq@savedpaths@xaddtikzpath@standard#1{\sseq@x@addto@macro\sseq@savedpaths{\@nx\sseq@tikzpath@object{#1}}}
\def\sseq@savedpaths@add@standard{\sseq@g@addto@macro\sseq@savedpaths}
\def\sseq@savedpaths@xadd@standard{\sseq@x@addto@macro\sseq@savedpaths}
\def\sseq@savedpaths@show@standard{\show\sseq@savedpaths}
\def\sseq@scope@savedpaths@add@standard#1{\sseq@savedpaths@add@standard{#1}\sseq@savedpaths@add@background{#1}}

\def\sseq@savedpaths@addclass@temporary#1{\sseq@x@addto@macro\sseq@tempsavedpaths{\@nx\sseq@class@draw@ifpage{#1}}}
\def\sseq@savedpaths@adddifferential@temporary#1{\sseq@x@addto@macro\sseq@tempsavedpaths{\@nx\sseq@differential@draw@ifpage{#1}}}
\def\sseq@savedpaths@addextension@temporary#1{\sseq@x@addto@macro\sseq@tempsavedpaths{\@nx\sseq@extension@draw@ifpage{#1}}}
\def\sseq@savedpaths@addstructline@temporary#1{\sseq@x@addto@macro\sseq@tempsavedpaths{\@nx\sseq@structline@draw@ifpage{#1}}}
\def\sseq@savedpaths@xaddtikzpath@temporary#1{\sseq@x@addto@macro\sseq@tempsavedpaths{#1}}
\def\sseq@savedpaths@add@temporary{\sseq@g@addto@macro\sseq@tempsavedpaths}
\def\sseq@savedpaths@xadd@temporary{\sseq@x@addto@macro\sseq@tempsavedpaths}
\def\sseq@savedpaths@show@temporary{\show\sseq@tempsavedpaths}
\def\sseq@scope@savedpaths@add@temporary#1{\sseq@savedpaths@add@temporary{#1}\sseq@savedpaths@add@background{#1}}


\def\sseq@savedpaths@addclass@background#1{\sseq@error@internal@n{illegal-add-background-mode}{\sseq@savedpaths@addclass}}
\def\sseq@savedpaths@adddifferential@background#1{\sseq@error@internal@n{illegal-add-background-mode}{\sseq@savedpaths@adddifferential}}
\def\sseq@savedpaths@addstructline@background#1{\sseq@error@internal@n{illegal-add-background-mode}{\sseq@savedpaths@addstructline}}
\def\sseq@savedpaths@addextension@background#1{\sseq@error@internal@n{illegal-add-background-mode}{\sseq@savedpaths@addextension}}
\def\sseq@savedpaths@xaddtikzpath@background#1{\sseq@x@addto@macro\sseq@savedbackgroundpaths{#1}}
\def\sseq@savedpaths@add@background{\sseq@g@addto@macro\sseq@savedbackgroundpaths}
\def\sseq@savedpaths@xadd@background{\sseq@x@addto@macro\sseq@savedbackgroundpaths}
\def\sseq@savedpaths@show@background{\show\sseq@savedbackgroundpaths}

\def\sseq@savedpaths@addclass@ignore#1{}
\def\sseq@savedpaths@adddifferential@ignore#1{}
\def\sseq@savedpaths@addstructline@ignore#1{}
\def\sseq@savedpaths@addextension@ignore#1{}
\def\sseq@savedpaths@xaddtikzpath@ignore#1{}
\def\sseq@savedpaths@add@ignore#1{}
\def\sseq@savedpaths@xadd@ignore#1{}
\def\sseq@savedpaths@show@ignore#1{}
\def\sseq@scope@savedpaths@add@ignore#1{}

\def\sseq@setsavedpaths#1{%
    \@xp\let\@xp\sseq@savedpaths@addclass       \csname sseq@savedpaths@addclass@#1\endcsname
    \@xp\let\@xp\sseq@savedpaths@adddifferential\csname sseq@savedpaths@adddifferential@#1\endcsname
    \@xp\let\@xp\sseq@savedpaths@addstructline  \csname sseq@savedpaths@addstructline@#1\endcsname
    \@xp\let\@xp\sseq@savedpaths@addextension  \csname sseq@savedpaths@addextension@#1\endcsname
    \@xp\let\@xp\sseq@savedpaths@xaddtikzpath  \csname sseq@savedpaths@xaddtikzpath@#1\endcsname
    \@xp\let\@xp\sseq@savedpaths@add            \csname sseq@savedpaths@add@#1\endcsname
    \@xp\let\@xp\sseq@savedpaths@xadd            \csname sseq@savedpaths@xadd@#1\endcsname
    \pgfutil@ifundefined{sseq@scope@savedpaths@add@#1}{}{%
        \@xp\let\@xp\sseq@savedpaths@show        \csname sseq@savedpaths@show@#1\endcsname
        \@xp\let\@xp\sseq@scope@savedpaths@add   \csname sseq@scope@savedpaths@add@#1\endcsname
    }%
}

% We don't want to add anything to paths except for tikz paths in background mode
% We need to draw those because they might affect the bounding box.
\def\sseq@setsavedpaths@draftmode{
    \sseq@setsavedpaths{ignore}     % Don't add stuff to paths
    % Only add tikzpaths if in background mode (so ignore in standard or temporary)
    \let\sseq@savedpaths@xaddtikzpath@standard\@gobble
    \let\sseq@savedpaths@xaddtikzpath@temporary\@gobble
    % Only update mode for tikzpath
    \def\sseq@setsavedpaths##1{\@xp\let\@xp\sseq@savedpaths@xaddtikzpath  \csname sseq@savedpaths@xaddtikzpath@##1\endcsname}
}

