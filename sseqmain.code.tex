%%
%% Package: spectralsequences v1.1.1 2017-09-16 2017-09-16
%% Author: Hood Chatham
%% Email: hood@mit.edu
%% Date: 2017-09-18
%% License: Latex Project Public License
%%
%% File: sseqmain.code.tex
%%
%%    Defines the main environments and commands
%%


%%%%%%%
%%%
%%%  Environment definitions
%%%
%%%%%%%
\def\sseq@usesavedpaths{%
    \sseq@savedpaths
    \sseq@clearchangestyles
    \sseq@tempsavedpaths
    \gdef\sseq@tempsavedpaths{}%
}

\sseq@addtostorelist\sseq@savedbackgroundpaths{}
\sseq@addtostorelist\sseq@restorestyles{}

\def\sseq@page{\protect\sseq@thepage}


\def\sseqdata{\@ifnextchar[{\sseqdata@}{\sseqdata@[]}}%}

\def\sseqdata@[#1]{%
    \edef\sseq@restorefont{\font@name}
    \selectfont\nullfont % Avoid space hazards!

    \def\par{}
%
    \ifx\pgfpictureid\@undefined\else %Uh-oh, we're inside tikz.
        \sseq@error@n{env-inside-tikz}{sseqdata}\@xp\sseq@breakdataenv
    \fi
    \ifsseq@inprogress % Uh-oh, we're already in a sseq environment. This is not good.
        \sseq@error@n{env-inside-sseq}{sseqdata}\@xp\sseq@breakdataenv
    \fi
    \sseq@inprogresstrue
    \def\sseq@sseqsetdirectory{/sseqpages}
    \sseq@keepchangestrue
    \sseq@storelist@setdefaults
    \let\sseq@xtickfn\@firstofone\let\sseq@ytickfn\@firstofone % These are default functions, which addtostorelist can't handle.
%
    \ifx\sseq@xmin@default\sseq@infinity
        \let\sseq@xminmax\sseq@xminmax@active
    \fi
    \ifx\sseq@ymin@default\sseq@infinity
        \let\sseq@yminmax\sseq@yminmax@active
    \fi
    \ifsseq@mathnodes
        \let\sseq@maybemathswitch=$\@gobble$
    \fi
%
    % copy commands into the user namespace.
    \sseq@installmacros
    \sseq@installmsghooks
    \sseq@install@xparse@Uarggrabber
    \def\xmin{\protect\sseq@xmin}\def\xmax{\protect\sseq@xmax}
    \def\ymin{\protect\sseq@ymin}\def\ymax{\protect\sseq@ymax}
%
    \sseq@modtikzcommands % Defer evaluation of tikz primitives
    \sseq@patchfor % patch \foreach to work with deferred macros
    \pgfkeys{/handlers/first char syntax/the character "/.initial=\sseq@handlequote} % install quotes parsing
%
%
    \pgfqkeys{/sseqpages/global/name only}{#1}
    \@ifundefined{sseq@thename}{\sseq@error{env-data-no-name}\sseq@breakdataenv}{}
    \sseq@obj@ifundef{exists}{
        \sseq@obj@gdef{exists}{}
        \pgfqkeys{/sseqpages/global}{{\sseq@thename} options/.code={}}
    }{
        \ifsseq@updateexisting\else
            \sseq@error@x{env-data-already-exist}{\sseq@thename}% this could be a warning
        \fi
        \sseq@restorestyles
        \sseq@getcmds

    }
    \sseq@setsavedpaths{standard}
%
    % the default option handler for /sseqpages/global adds to \sseq@currentoptionpath options
    \let\sseq@currentoptionpath\sseq@thename
    \pgfqkeys{/sseqpages/global}{#1}
    \ifx\sseq@targetx\pgfutil@empty\else
        \sseq@hasdegreetrue
    \fi
    \sseq@eval{
        \global\let\@xp\@nx\csname pgfk@/sseqpages/global/{\sseq@currentoptionpath} options/.@cmd\endcsname
            \@xp\@nx\csname pgfk@/sseqpages/global/{\sseq@currentoptionpath} options/.@cmd\endcsname
    }
    \sseq@savedpaths@xadd{\@nx\sseq@style@object{\sseq@stylelist}}
    \sseq@savedpaths@add{\begingroup} % prevent some of the two-pass user macros horrors
    \let\sseq@error@setup\sseq@error@setup@withinfo
}

% Just store all the settings
\def\endsseqdata{
    \let\sseq@error@setup\sseq@error@setup@noinfo
    \sseq@checkend{sseqdata}
    \sseq@savedpaths@add{\endgroup} % prevent some of the two-pass user macros horrors
    \xdef\sseq@restorestyles{\sseq@stylelist}
    \sseq@storecmds
    \ignorespacesafterend
    \sseq@breakpoint
}

\def\sseqpage{\@ifnextchar[{\sseqpage@}{\sseqpage@[]}} %]

\def\sseqpage@[#1]{%
    \edef\sseq@restorefont{\font@name}
    \selectfont\nullfont % Avoid space hazards!
    \def\par{}
%
    \ifx\pgfpictureid\@undefined\else %Uh-oh, we're inside tikz.
        \sseq@error@n{env-inside-tikz}{sseqpage}\@xp\sseq@breakpageenv
    \fi
    \ifsseq@inprogress % Uh-oh, we're already in a sseq environment. This is not good.
        \sseq@error@n{env-inside-sseq}{sseqpage}\@xp\sseq@breakpageenv
    \fi
    \sseq@inprogresstrue
    \sseq@ispageenvtrue
    \def\sseq@sseqsetdirectory{/sseqpages}
%
%    \ifsseq@mathnodes
%        \let\sseq@maybemathswitch=$\@gobble$
%    \fi
%
    % Copy commands into user namespace
    \sseq@installmacros
    \sseq@installmsghooks
    \sseq@install@xparse@Uarggrabber
    \def\xmin{\protect\sseq@xmin}\def\xmax{\protect\sseq@xmax}
    \def\ymin{\protect\sseq@ymin}\def\ymax{\protect\sseq@ymax}
    \sseq@modtikzcommands % Redefine tikz primitives to store themselves in the appropriate place
    \sseq@patchfor % Change foreach to record it's variable values in the appropriate places
    \pgfkeys{/handlers/first char syntax/the character "/.initial=\sseq@handlequote}% install quotes parsing
%
    \gdef\sseq@pagecleanup{}% For undoing local option settings
%
    \def\sseq@thepage{0}% page defaults to 0
    \sseq@thepagecount=\z@
    \pgfqkeys{/sseqpages/global/name only}{#1}% Get the name
    \sseq@tempiftrue
    \ifsseq@hasname
        \sseq@obj@ifundef{exists}{
            \sseq@error@x{env-page-not-exist}{\sseq@thename} % This could be a warning
            \sseq@obj@gdef{exists}{}
            \let\sseq@currentoptionpath\sseq@thename
            \pgfqkeys{/sseqpages/global}{{\sseq@thename} options/.code={}}
            \sseq@keepchangestrue
            \sseq@keepglobaloptionstrue
            \sseq@setsavedpaths{standard}
        }{
            \sseq@restorestyles
            \sseq@getcmds
            \ifsseq@keepchanges
                \sseq@setsavedpaths{standard}
            \else
                \sseq@setsavedpaths{temporary}
            \fi
            \sseq@tempiffalse
            \def\sseq@currentoptionpath{this page}
            \pgfqkeys{/sseqpages/global}{{this page} options/.code={},#1}
        }
    \else
        \edef\sseq@thename{temp@sseq@number@\the\sseq@anonsseqcount}
        \global\advance\sseq@anonsseqcount1
        \sseq@setsavedpaths{temporary}
    \fi
    \ifsseq@tempif % Either the spectral sequence has no name, or it has a name but no such existed.
        \sseq@storelist@setdefaults
        \sseq@savedpaths@xadd{\sseq@stylelist}
        \let\sseq@currentoptionpath\sseq@thename
        \pgfqkeys{/sseqpages/global}{{\sseq@thename} options/.code={},{this page} options/.code={},#1}
        \sseq@eval{
            \global\let\@xp\@nx\csname pgfk@/sseqpages/global/{\sseq@currentoptionpath} options/.@cmd\endcsname
                \@xp\@nx\csname pgfk@/sseqpages/global/{\sseq@currentoptionpath} options/.@cmd\endcsname
        }
    \fi
%
    \ifsseq@specifiedxrange
        \let\sseq@xminmax\@gobbletwo
    \else
        \let\sseq@xminmax\sseq@xminmax@active
    \fi
    \ifsseq@specifiedyrange
        \let\sseq@yminmax\@gobbletwo
    \else
        \let\sseq@yminmax\sseq@yminmax@active
    \fi
%
    \sseq@savedpaths@xadd{\sseq@stylelist}
    \sseq@savedpaths@add{\begingroup} % prevent some of the two-pass user macros horrors
    \ifx\sseq@targetx\pgfutil@empty\else
        \sseq@hasdegreetrue
    \fi
    \let\sseq@error@setup\sseq@error@setup@withinfo
}

% Here we do all of the work to actually print the page
\def\endsseqpage{
    \let\sseq@error@setup\sseq@error@setup@noinfo
    \sseq@checkend{sseqpage}
    \sseq@savedpaths@add{\endgroup} % prevent some of the two-pass user macros horrors
%
    \sseq@xscalecm=\sseq@xscale cm
    \sseq@yscalecm=\sseq@yscale cm
%
% check that range is okay
%
    \ifx\sseq@xmin\sseq@infinity\sseq@error{no-range}\def\sseq@xmin{0}\def\sseq@xmax{0}\@xp\sseq@break\fi % could be warning
    \ifx\sseq@ymin\sseq@infinity\sseq@error{no-range}\def\sseq@ymin{0}\def\sseq@ymax{0}\@xp\sseq@break\fi
%
    \ifnum\numexpr\sseq@xmax-\sseq@xmin\relax>\numexpr\maxdimen/\dimexpr1pt\relax-1\relax
        \sseq@error@xx{range-super-overflow}{x}
            {from \sseq@xmin\space to \sseq@xmax\space (xmax - xmin = \the\numexpr\sseq@xmax-\sseq@xmin)}
        \@xp\sseq@break
    \fi
    \ifnum\numexpr\sseq@ymax-\sseq@ymin\relax>\numexpr\maxdimen/\dimexpr1pt\relax-1\relax
        \sseq@error@xx{range-super-overflow}{y}
            {from \sseq@ymin\space to \sseq@ymax\space (ymax - ymin = \the\numexpr\sseq@ymax-\sseq@ymin)}
        \@xp\sseq@break
    \fi
%
    \sseq@tempdimen = \dimexpr \textwidth -\sseq@clip@padding@right - \sseq@xaxis@end@extend - \sseq@xaxis@start@extend - \sseq@yaxisgap \relax
    \sseq@tempcount = \numexpr \sseq@tempdimen /\sseq@xscalecm \relax
    \ifnum \numexpr\sseq@xmax-\sseq@xmin + 1 > \sseq@tempcount
      \pgfmathparse{\sseq@tempdimen/1cm/(\sseq@xmax-\sseq@xmin+1)}
      \sseq@error@xxxx{range-overflow}{x}
            {from \sseq@xmin\space to \sseq@xmax\space (xmax - xmin = \the\numexpr\sseq@xmax-\sseq@xmin)}
            {\the\sseq@tempcount}% max x range
            {\pgfmathresult}% max xscale
        \@xp\sseq@break
    \fi
%
    \sseq@tempdimen = \dimexpr \textheight - \sseq@clip@padding@top - \sseq@yaxis@end@extend - \sseq@yaxis@start@extend - \sseq@xaxisgap \relax
    \sseq@tempcount = \numexpr \sseq@tempdimen /\sseq@yscalecm \relax
    \ifnum \numexpr\sseq@ymax-\sseq@ymin +1 > \sseq@tempcount
        \pgfmathparse{\sseq@tempdimen/1cm/(\sseq@ymax-\sseq@ymin+1)}
        \sseq@error@xxxx{range-overflow}{y}
            {from \sseq@ymin\space to \sseq@ymax\space (ymax - ymin = \the\numexpr\sseq@ymax-\sseq@ymin)}
            {\the\sseq@tempcount}% max x range
            {\pgfmathresult}% max xscale
        \@xp\sseq@break
    \fi
%
    % Really this could be anything between -\sseq@xmin and -\sseq@xmax and it wouldn't matter. We'll split the difference.
    \sseq@xoffset=\numexpr\sseq@intdivceiling{\numexpr-\sseq@xmin-\sseq@xmax\relax}{\numexpr4*\sseq@xgridstep\relax}*(2*\sseq@xgridstep)\relax
    \sseq@yoffset=\numexpr\sseq@intdivceiling{\numexpr-\sseq@ymin-\sseq@ymax\relax}{\numexpr4*\sseq@ygridstep\relax}*(2*\sseq@ygridstep)\relax
%
%
    \tikzset{
        z=0pt,
        every path/.style={},
        every node/.style={}
    }
    \let\errmessage\sseq@errmessage@std
    \sseq@setlayoutparameters
    \begin{tikzpicture}[/sseqpages/global/{default} options,\ifsseq@hasname/sseqpages/global/{\sseq@thename} options\fi,/sseqpages/global/{this page} options]%
    \sseq@sseqstyle@globalpart
    \sseq@globaldetonetransform
    \sseq@globalscaletransform
%
    \let\sseq@transform\sseq@error@illegaltransform % Disallow most coordinate transforms
    \let\sseq@shifttransform\sseq@checkshifttransform % Allow shifts as long as they are by integers and have no units.
%
    \sseq@setupclip
%
    \sseq@patchtikzcoords
%
    \pgfonlayer{background}%
    % Axes labels only occur once per page, so the cost of typesetting them with tikz doesn't matter ... THANK GOD
    \ifx\sseq@title\pgfutil@empty\else
        \ifsseq@drawtitle
            \sseq@eval{\@nx\path (0,-\sseq@yoffset) node[
                /sseqpages/global/every title,
                above=\sseq@topclippadding+5pt
                ] {\unexpanded\expandafter{\sseq@title}};
            }
        \fi
    \fi
    % Don't touch the shifts -- they are very persnickety.
    \ifx\sseq@xaxislabel\pgfutil@empty\else\ifsseq@drawxaxislabel
        \sseq@eval{\noexpand\node[
            align=center,
            /sseqpages/global/every x axis label,
            below,
            shift={\sseq@xlabelposition},
        ] {\unexpanded\expandafter{\sseq@xaxislabel}};}
    \fi\fi
    \ifx\sseq@yaxislabel\pgfutil@empty\else\ifsseq@drawyaxislabel
        \sseq@eval{\noexpand\node[
            align=center,
            /sseqpages/global/every y axis label,
            above,
            rotate=90,
            shift={\sseq@ylabelposition},
        ] {\unexpanded\expandafter{\sseq@yaxislabel}};}
    \fi\fi
%
    \sseq@grid
    \sseq@handlexaxis % Draw the axis with pgf primitives. These are defined in sseqdrawing.code.tex
    \sseq@handleyaxis
    \endpgfonlayer
%
    \ifsseq@mathnodes
        \tikzset{execute at begin node=$,execute at end node=$}%
    \fi
%
    \begingroup
    \sseq@savedbackgroundpaths
    \endgroup
%
    \sseq@useclip
%
    \ifsseq@drawdifferentials\else
        \let\sseq@differential@object\@gobble
    \fi
    \ifsseq@drawstructlines\else
        \let\sseq@structline@object\@gobble
    \fi
%
    \sseq@usesavedpaths
%
    \ifsseq@keepchanges
        \xdef\sseq@restorestyles{\sseq@stylelist}
    \fi
    \end{tikzpicture}%
    \ifsseq@keepchanges
        \sseq@storecmds   % store changes
    \else
        \sseq@pagecleanup % undo any changes specific to this page
    \fi
    \sseq@breakpoint
    \ignorespacesafterend
}

% Shorthand for sseqpage with empty body
\def\printpage[#1]{%
    \begin{sseqpage}[#1]
    \end{sseqpage}%
}


\def\sseqcopypage#1#2#3{
    \begingroup
    \def#1{\begingroup}
    \def\sseq@copymacro{#1}
    \def\sseq@thename{#2}
    \sseq@storelist@get\sseq@savedpaths
    \def\sseq@thepage{#3}
    \sseq@thepagecount=#3\relax
    \def\sseq@beginscope@object{\sseq@g@addto@macro#1{\sseq@savedpaths@add{\sseq@beginscope@object}}}
    \def\sseq@endscope@object{\sseq@g@addto@macro#1{\sseq@savedpaths@add{\sseq@endscope@object}}}
    \def\sseq@scope@object##1{\sseq@g@addto@macro#1{\sseq@savedpaths@add{\sseq@scope@object{##1}}}}
    \def\sseq@style@object##1{\sseq@g@addto@macro#1{\sseq@savedpaths@add{\sseq@style@object{##1}}}}
    \let\sseq@class@drawnode\sseq@copypage@addclass
    \let\sseq@differential@object\@gobble
    \let\sseq@structline@draw\sseq@copypage@addstructline
    \let\sseq@tikzpath@object\@gobble
    \let\sseq@pgfkeysdocommand\@gobbletwo
    \sseq@savedpaths
    \sseq@g@addto@macro#1{\endgroup}
    \endgroup
}

\def\sseq@copypage@addclass#1{
    \@xp\sseq@x@addto@macro\sseq@copymacro{\@nx\sseq@class@copy{\sseq@thename}{#1}}
}

\def\sseq@copypage@addstructline#1{
    \@xp\sseq@x@addto@macro\sseq@copymacro{\@nx\sseq@structline@copy{\sseq@thename}{#1}}
}


%%% Scope
\newcount\sseq@scope@savestackid
\def\sseq@scope@secondpasstoks{}
\def\sseq@scope{\@ifnextchar[{\sseq@scope@}{\sseq@scope@[]}}%]
\def\sseq@scope@[#1]{%
    \sseq@scope@savedpaths@add{\sseq@beginscope@object}%
    % This is a funky hack to fix a problem with shifts. On tmfass, if we don't handle shifts specially, we actually put so much stuff into
    % this \sseq@scope@secondpasstoks macro that we exceed TeX's upper bound for the maximum total macro memory. Or something like that -- I don't understand the error,
    % but it definitely only depends on how much stuff there is overall and on there being lots of scopes, not on any particular part of the diagram.
    % Anyways, to avoid this excessive storage demand, we don't put the shifts into the scope toks and apply them tons of times, we just do them once.
    \let\sseq@options@scopeshiftsecondpass\sseq@scope@savedpaths@add
    \sseq@processoptions{scope}{#1}% outputs into \sseq@savedoptioncode
    \edef\sseq@scope@secondpasstoks{\unexpanded\@xp{\sseq@scope@secondpasstoks}\unexpanded\@xp{\sseq@savedoptioncode}}
    \sseq@eval{\@nx\sseq@scope@savedpaths@add{\@nx\sseq@scope@object{\sseq@scope@toks{\unexpanded\@xp{\sseq@scope@secondpasstoks}}}}}%
    %\advance\sseq@scope@savestackid\@ne
    %\sseq@savestack@name{scope@save\the\sseq@scope@savestackid}
}
\def\sseq@endscope{%
    \sseq@scope@savedpaths@add{\sseq@endscope@object}
    %\sseq@restorestack@name{scope@save\the\sseq@scope@savestackid
}




\ifsseq@draftmode % Okay, have to dummy out all the main commands.
    % Only add tikz background paths to the savedpaths. We need to draw background paths because they might change the bounding box!
    \sseq@setsavedpaths@draftmode
%
    % All we want to do here is record the range. Have to process options in order to figure out where it is.
    \sseq@DeclareDocumentCommand\class{or()}{%
        \begingroup
        \sseq@loadinputline
        \IfNoValueT{#2}{\sseq@break}%
        \sseq@xsetthiscall{\string\class\IfNoValueF{#1}{[\unexpanded{#1}]}\unexpanded{(#2)}}%
        \sseq@options@firstpassmode
        \the\sseq@sseqstyle
        \the\sseq@classstyle
        \sseq@options@normalmode
        \IfNoValueF{#1}{%
            \sseq@processoptions{class}{#1}% Outputs result into \sseq@savedoptioncode
        }%
        \sseq@cparsecoord{#2}{}%
        \sseq@xminmax{\xcoord}{\ycoord}%
        \sseq@yminmax{\xcoord}{\ycoord}%
        \sseq@pushstack@\coord % also need the class stack to calculate other \class commands.
        \sseq@cleanup@obj{partcoord.(\partialcoord).numnodes}%
        \ifnum\nodenum=\@ne
            \sseq@obj@gdef{partcoord.(\partialcoord).numnodes}{0}%
        \fi
        \sseq@obj@xdef{partcoord.(\partialcoord).numnodes}{\nodenum}%
        \edef\classname{class.\coord[0]}%
        \ifx\sseq@class@tag\pgfutil@empty\else
            \sseq@obj@ifdef{partcoord.(\partialcoord).tag.\sseq@class@tagprefix\sseq@class@tag}{%
                \sseq@error@xx{class-tag-already-defined}{\sseq@class@tagprefix\sseq@class@tag}{(\partialcoord)}% could be warning
            }{}%
            \sseq@cleanup@obj{partcoord.(\partialcoord).tag.\sseq@class@tagprefix\sseq@class@tag}%
            \sseq@obj@xdef{partcoord.(\partialcoord).tag.\sseq@class@tagprefix\sseq@class@tag}{\nodenum}%
        \fi
        \ifx\sseq@class@name\pgfutil@empty\else
            \sseq@cleanup@obj{class.\classname.name}
            \sseq@cleanup@obj{class.namedclass.\sseq@class@name}
            \sseq@namehandler
            \sseq@obj@xdef{class.\classname.name}{\sseq@class@name}
            \sseq@obj@xdef{class.namedclass.\sseq@class@name}{\coordnopar}
        \fi
        \sseq@breakpoint
        \endgroup
    }

    % Now the rest of these can be no-ops
    \sseq@DeclareDocumentCommand\classoptions{od()}{}
    \sseq@DeclareDocumentCommand\replaceclass{od()}{}
    \def\sseq@d@main#1#2#3#4#5{} % makes both \d and \doptions do nothing
    \sseq@DeclareDocumentCommand\structline{od()d()}{}
    \sseq@DeclareDocumentCommand\structlineoptions{od()d()}{}
    \sseq@DeclareDocumentCommand\circleclasses{or()r()}{}
    \endinput
\fi


\protected\def\sseq@xcoord{}\protected\def\sseq@ycoord{}

%%%%%
%%
%% The main commands
%%
%%%%%
%%% \class, \classoptions, \replaceclass
\sseq@DeclareDocumentCommand\class{or()}{%
    \begingroup
    \sseq@loadinputline
    \IfNoValueT{#2}{\sseq@break}
    \sseq@xsetthiscall{\string\class\IfNoValueF{#1}{[\unexpanded{#1}]}\unexpanded{(#2)}}%
    \sseq@options@firstpassmode
    \the\sseq@sseqstyle
    \the\sseq@classstyle
    \the\sseq@scope@toks
    \sseq@options@normalmode
    \IfNoValueF{#1}{%
        \sseq@processoptions{class}{#1}% Outputs result into \sseq@savedoptioncode
    }%
%
    \@ifundefined{sseq@class@page}{}{\sseq@error@xxx{cmdoptions-only}{page}      {\string\class}{\string\classoptions}}% could be warnings
    \@ifundefined{sseq@gen} {}{\sseq@error@xxx{cmdoptions-only}{generation}{\string\class}{\string\classoptions}}
%
    \sseq@cparsecoord{#2}{}%
    \ifnum\csname sseq@\sseq@classpattern @maxclasses\endcsname<\nodenum\relax
        \sseq@error@xxxx{classpattern-too-many-classes}{(\partialcoord)}{\nodenum}{\sseq@classpattern}{\@xp\the\csname sseq@\sseq@classpattern @maxclasses\endcsname}
        \sseq@breakfi
    \fi
%%%% No more breaks after this point
    \sseq@processlabels % This sets up the label nodes, now that we know the position of the class (I think that's what we needed?)
    \sseq@class@main % We segment off the main part of the class so that we can call it separately from the copy method
    \sseq@breakpoint
    \endgroup
}
%
\def\sseq@class@main{
    \ifx\sseq@classinsert\empty
        % by default, just insert to the right of previous classes
        \edef\sseq@classinsert{\nodenum}
    \else
        \ifnum\sseq@classinsert<\z@
            % negative 1 is where we would normally insert things
            \edef\sseq@classinsert{\the\numexpr\nodenum+\sseq@classinsert+1}
        \fi
        \ifnum\sseq@classinsert<\@ne
            \def\sseq@classinsert{1}
        \fi
        \ifnum\sseq@classinsert>\nodenum\relax
            \edef\sseq@classinsert{\nodenum}
        \fi
        \ifnum\nodenum>\@ne
            \@xp\sseq@class@adjustclasses\@xp{\partialcoord} % Adjust all of the indices of the classes to the right of the one we're inserting
        \fi
    \fi
%
%
    \sseq@xminmax{\xcoord}{\ycoord}%
    \sseq@yminmax{\xcoord}{\ycoord}%
    \sseq@cleanup@obj{partcoord.(\partialcoord).numnodes}%
    \sseq@obj@xdef{partcoord.(\partialcoord).numnodes}{\nodenum}%
    \edef\classname{class.\coord[0]}%
    \ifx\sseq@class@tag\pgfutil@empty\else
        \sseq@obj@ifdef{partcoord.(\partialcoord).tag.\sseq@class@tagprefix\sseq@class@tag}{
            \sseq@error@xx{class-tag-already-defined}{\sseq@class@tagprefix\sseq@class@tag}{(\partialcoord)}% could be warning
        }{}
        \sseq@cleanup@obj{class.\coord.tag}
        \sseq@obj@xdef{class.\coord.tag}{\sseq@class@tagprefix\sseq@class@tag}
        \sseq@cleanup@obj{partcoord.(\partialcoord).tag.\sseq@class@tagprefix\sseq@class@tag}
        \sseq@obj@xdef{partcoord.(\partialcoord).tag.\sseq@class@tagprefix\sseq@class@tag}{\nodenum}%
    \fi
    \ifx\sseq@class@name\pgfutil@empty\else
        \sseq@cleanup@obj{class.\classname.name}
        \sseq@cleanup@obj{class.namedclass.\sseq@class@name}
        \sseq@namehandler
        \sseq@obj@xdef{class.\classname.name}{\sseq@class@name}
        \sseq@obj@xdef{class.namedclass.\sseq@class@name}{\coordnopar}
    \fi
    \ifcsname sseq@class@showname\endcsname
        \sseq@eval{\@nx\sseq@handleclassquotes{\sseq@obj{class.\classname.name}}{\sseq@class@showname}}
    \fi
    \sseq@pushstack@\coord
%
%
    \sseq@obj@gdef{class.\coord.num}{0}% For \replaceclass
    \sseq@obj@xdef{\classname.page}{\sseq@infinity}%
    \sseq@obj@xdef{class.\coord.n}{\sseq@classinsert}
    \ifx\sseq@familylist\pgfutil@empty\else
        \sseq@obj@xodef{\classname.familylist}{\sseq@familylist}
    \fi
    \ifx\sseq@savedoptioncode\pgfutil@empty\else
        \sseq@obj@xodef{\classname.options}{\sseq@savedoptioncode}%
    \fi
    \ifx\sseq@classnodetext\pgfutil@empty\else
        \sseq@obj@xodef{\classname.nodetext}{\sseq@classnodetext}
    \fi
    \ifx\sseq@classnodetextoptions\pgfutil@empty\else
        \sseq@obj@xodef{\classname.nodetext.options}{\sseq@classnodetextoptions}
    \fi
    \ifx\sseq@classlabelnodes\pgfutil@empty\else
        \sseq@obj@xodef{\classname.labelnodes}{\sseq@classlabelnodes}%
    \fi
    \ifsseq@needstikz
        \sseq@obj@xdef{\classname.needstikz}{\sseq@needstikztrue}%
    \fi
    \@ifundefined{sseq@tempoffset}{}{%
        \tikz@scan@one@point\sseq@setoffset\sseq@tempoffset
    }%
    \ifx\sseq@tooltip\pgfutil@empty\else
        \sseq@obj@protectedxdef{\classname.tooltip}{\sseq@tooltip}
    \fi
%
    \sseq@savedpaths@addclass{class.\coord}%
}
\def\sseq@setoffset#1{\sseq@obj@gdef{\classname.offset}{\pgftransformshift{#1}}}

\def\sseq@class@copy#1#2{
    \begingroup
    \let\sseq@thename@saved\sseq@thename
    \def\sseq@thename{#1}
    \sseq@lettoobj\sseq@familylist{#2.familylist}
    \sseq@lettoobj\sseq@savedoptioncode{#2.options}
    \sseq@lettoobj\sseq@classnodetext{#2.nodetext}
    \sseq@lettoobj\sseq@classnodetextoptions{#2.nodetext.options}
    \sseq@lettoobj\sseq@classlabelnodes{#2.labelnodes}
    \sseq@obj{#2.needstikz}
    \sseq@lettoobj\sseq@tooltip{#2.tooltip}
    \sseq@class@getparts#2.
    \edef\sseq@class@tag{copiedfrom\sseq@thisposnum}
    \let\sseq@thename\sseq@thename@saved
    \sseq@eval{\@nx\sseq@cparsecoord{\the\sseq@tempx,\the\sseq@tempy}}
    \let\sseq@pushstack@\@gobble
    \sseq@class@main
    \endgroup
}


% This is for use with "insert" key. Add one to offset index of each class later than the one currently being inserted
% Takes the partial coordinate "x,y" in #1, and the position that the new class is being inserted into in \sseq@classinsert.
\def\sseq@class@adjustclasses#1{
    \sseq@tempcount=\z@
    \sseq@tempcountb=\sseq@obj{partcoord.(#1).numnodes}
    \loop
        \advance\sseq@tempcount\@ne
        \ifnum\sseq@obj{class.(#1,\the\sseq@tempcount).n}<\sseq@classinsert\relax\else
            \sseq@cleanup@obj{class.(#1,\the\sseq@tempcount).n}
            \sseq@eval{\@nx\sseq@obj@inc{class.(#1,\the\sseq@tempcount).n}} % unfortunately, \sseq@obj@inc uses \sseq@tempcount so we need this eval
        \fi
    \ifnum\sseq@tempcount<\sseq@tempcountb\repeat

}

\sseq@DeclareDocumentCommand\classoptions{od()}{%
    \begingroup
    \sseq@loadinputline
    \sseq@xsetthiscall{\string\classoptions\IfNoValueF{#1}{[\unexpanded{#1}]}\unexpanded{(#2)}}%
    \IfNoValueT{#1}{\sseq@error@x{cmdoptions-without-options}{\string\classoptions}\sseq@break}% could be warning
    \sseq@options@firstpassmode
    \the\sseq@sseqstyle
    \the\sseq@classstyle
    \the\sseq@scope@toks
    \sseq@options@normalmode
    \let\sseq@processlabels\sseq@processlabels@default % Don't double dip on labels from the scope or classstyle
    \sseq@processoptions{class}{#1}% Outputs result into \sseq@savedoptioncode
%
    \IfNoValueTF{#2}{
        \def\sseq@coord{\lastclass0}
    }{
        \def\sseq@coord{#2}
    }
    \sseq@parsecoord\coord{\sseq@coord}{\string\classoptions}%
%
    \sseq@processlabels
%
    \let\partialcoord\coord@partial
    \edef\sseq@classnum{\sseq@obj@ifdef{class.\coord.num}{\sseq@obj{class.\coord.num}}{0}}
    \@ifundefined{sseq@class@page}{%
        \@ifundefined{sseq@gen}{
            \@ifundefined{sseq@thepage}{
                \let\sseq@gen\sseq@classnum
            }{
                \sseq@obj@pagetogen{class.\coord}{\sseq@thepage} % stores in \sseq@gen
            }
            \let\sseq@genmax\sseq@gen
        }{
            \ifnum\sseq@gen<\z@
                \edef\sseq@gen{\the\numexpr\sseq@classnum+\sseq@gen+1}
            \fi
            \@ifundefined{sseq@genmax}{\let\sseq@genmax\sseq@gen}{
                \edef\sseq@genmax{\the\numexpr\sseq@classnum+\sseq@genmax+1}
            }
        }
    }{
        \@ifundefined{sseq@gen}{}{\sseq@error{classoptions-page-gen}}% This could be a warning
        \sseq@class@getgen{\sseq@class@page}
        \@ifundefined{sseq@class@pagemax}{
            % If no max is present, apply options to all pages starting at specified minimum
            \let\sseq@genmax\sseq@classnum%
            \let\sseq@class@pagemax\sseq@infinity % ensure if "page" is defined, "pagemax" is defined too
        }{
            \let\sseq@genmin\sseq@gen
            \sseq@class@getgen{\sseq@class@pagemax}
            \let\sseq@genmax\sseq@gen
            \let\sseq@gen\sseq@genmin
        }
        \edef\sseq@class@page{\the\numexpr\sseq@class@page-\@ne}
        \edef\sseq@class@pagemax{\the\numexpr\sseq@class@pagemax+\@ne}
    }
    \ifnum\sseq@gen=\m@ne
        \let\sseq@gen\sseq@classnum
    \fi
    \edef\classname{class.\coord[\sseq@gen]}
    \ifsseq@keepchanges\else
        \ifnum\sseq@thepagecount>\sseq@obj{\classname.page}\relax
            \sseq@error@xxx{classoptions-already-dead}{\sseq@gen}{\coordname}{\sseq@obj{\classname.page}}
            \sseq@breakfifi
        \fi
        \ifnum\sseq@gen>\z@
            \ifnum\sseq@thepagecount>\sseq@obj{class.\coord[\the\numexpr\sseq@gen-\@ne].page}\relax\else
                \sseq@error@xx{classoptions-not-yet-born}{\sseq@gen}{\coord}{\sseq@class@page}%
                \sseq@breakfififi
            \fi
        \fi
    \fi{}
    \ifx\sseq@class@tag\pgfutil@empty\else
        \sseq@obj@ifdef{partcoord.(\partialcoord).tag.\sseq@class@tagprefix\sseq@class@tag}{%
            \sseq@error@nn{class-tag-already-defined}{\sseq@class@tagprefix\sseq@class@tag}{\sseq@partialcoord}% could be warning
        }%
        \sseq@cleanup@obj{class.\coord.tag}%
        \sseq@obj@xdef{class.\coord.tag}{\sseq@class@tagprefix\sseq@class@tag}%
        \sseq@cleanup@obj{partcoord.(\partialcoord).tag.\sseq@class@tagprefix\sseq@class@tag}%
        \sseq@obj@xdef{partcoord.(\partialcoord).tag.\sseq@class@tagprefix\sseq@class@tag}{\sseq@index}%
    \fi
    \ifx\sseq@class@name\pgfutil@empty\else
        \sseq@obj@ifdef{class.\classname.name}{
            \sseq@error@nn{classoptions-class-already-named}{\coord}{\sseq@obj{class.\classname.name}}%could be warning
        }{}
        \sseq@cleanup@obj{class.\classname.name}
        \sseq@cleanup@obj{class.namedclass.\sseq@class@name}
        \sseq@namehandler
        \sseq@obj@xdef{class.\classname.name}{\sseq@class@name}
        \sseq@obj@xdef{class.namedclass.\sseq@class@name}{\coordnopar}
    \fi
    \sseq@tempcount=\sseq@gen\relax
    \advance\sseq@tempcount\m@ne
    \loop
        \advance\sseq@tempcount\@ne
        \edef\classname{class.\coord[\the\sseq@tempcount]}
        \sseq@cleanup@obj{\classname.options}
        \sseq@classoptions@setup@pagerange
        \sseq@classoptions@obj@addto@withpagerange{\classname.options}{\sseq@savedoptioncode}
    %
        \ifx\sseq@classnodetext\pgfutil@empty\else
            \sseq@cleanup@obj{\classname.nodetext}
            \@ifundefined{sseq@class@page}{}{
                \sseq@obj@xdef{\classname.nodetext}{
                    \@nx\ifnum\@nx\sseq@thepagecount<\the\numexpr\sseq@class@page+1\relax\relax
                    \unexpanded\@xptwo{\csname \sseq@obj@fullname{\classname.nodetext}\endcsname\fi}
                }
            }
            \sseq@classoptions@obj@addto@withpagerange{\classname.nodetext}{\sseq@classnodetext}
        \fi
        \ifx\sseq@classnodetextoptions\pgfutil@empty\else
            \sseq@cleanup@obj{\classname.nodetext.options}
            \sseq@classoptions@obj@addto@withpagerange{\classname.nodetext.options}{\sseq@classnodetextoptions}
        \fi
        \ifx\sseq@classlabelnodes\pgfutil@empty\else
            \sseq@cleanup@obj{\classname.labelnodes}
            \sseq@classoptions@obj@addto@withpagerange{\classname.labelnodes}{\sseq@classlabelnodes}
        \fi
        \ifx\sseq@familylist\pgfutil@empty\else
            \sseq@cleanup@obj{\classname.familylist}
            \sseq@classoptions@obj@addto@withpagerange{\classname.familylist}{\sseq@familylist}
        \fi
        \@ifundefined{sseq@tempoffset}{}{
            \sseq@cleanup@obj{\classname.offset}
            \tikz@scan@one@point\sseq@setoffset\sseq@tempoffset
        }
    \ifnum\sseq@tempcount<\sseq@genmax\repeat
    \sseq@breakpoint
    \endgroup
}
\def\sseq@classoptions@setup@pagerange{%
    \@ifundefined{sseq@class@page}{
        \def\sseq@classoptions@obj@addto@withpagerange@{\sseq@obj@gaddto{\sseq@tempobj}}
    }{
        \def\sseq@temp##1{##1}
        \ifnum\sseq@tempcount=\sseq@gen\relax
            \sseq@eval{\@nx\pretocmd\@nx\sseq@temp{\@nx\ifnum\sseq@thepagecount>\sseq@class@page\relax}{}{%
                \@nx\sseq@error@x{this-shouldnt-happen}{file main, function \string\sseq@classoptions@setup@pagerange}
            }}%
            \apptocmd\sseq@temp{\fi}{}{\sseq@error@x{this-shouldnt-happen}{file main, function \string\sseq@classoptions@setup@pagerange}}
        \fi
        \ifnum\sseq@tempcount=\sseq@genmax\relax
            \sseq@eval{\@nx\pretocmd\@nx\sseq@temp{\@nx\ifnum\sseq@thepagecount<\sseq@class@pagemax\relax}{}{\@nx\sseq@error@internal{}}}
            \apptocmd\sseq@temp{\fi}{}{\sseq@error@internal{this-shouldnt-happen}{file main, function \string\sseq@classoptions@setup@pagerange}}
        \fi
        \@xp\def\@xp\sseq@temp\@xp##\@xp1\@xp{\@xp{\sseq@temp{##1}}}
        \pretocmd\sseq@temp{\sseq@obj@gaddto{\sseq@tempobj}}{}{\sseq@error@internal{this-shouldnt-happen}{file main, function \string\sseq@classoptions@setup@pagerange}}
        \let\sseq@classoptions@obj@addto@withpagerange@\sseq@temp
    }
}
\def\sseq@classoptions@obj@addto@withpagerange#1#2{%
    \def\sseq@tempobj{#1}%
    \@xp\sseq@classoptions@obj@addto@withpagerange@\@xp{#2}%
}
\def\sseq@class@getgen#1{
    \bgroup
        \ifnum\sseq@classnum>\z@
            \sseq@eval{\unexpanded{\sseq@obj@pagetogen{class.\coord}}{#1}} % stores in \sseq@gen
        \else
            \def\sseq@gen{0}
        \fi
        \sseq@smuggle@macro\sseq@gen
    \egroup
}
\sseq@DeclareDocumentCommand\replaceclass{od()}{%
    \begingroup
    \sseq@loadinputline
    \sseq@xsetthiscall{\string\replaceclass\IfNoValueF{#1}{[\unexpanded{#1}]}\unexpanded{(#2)}}
    \sseq@options@firstpassmode
    \the\sseq@sseqstyle
    \the\sseq@classstyle
    \the\sseq@scope@toks
    \sseq@options@normalmode
    \IfNoValueF{#1}{%
        \sseq@processoptions{class}{#1}% Outputs result into \sseq@savedoptioncode
    }%
%
    \IfNoValueTF{#2}{
        \def\sseq@coord{\lastclass0}
    }{
        \def\sseq@coord{#2}
    }
    \sseq@parsecoord\coord{\sseq@coord}{\string\classoptions}%
    \sseq@processlabels
%
    \sseq@obj@ifundef{class.\coord.num}{\sseq@error@x{replaceclass-no-class}{\coordname}\sseq@break}{}
    \sseq@ifdead\coord{}{\sseq@error@x{replaceclass-not-dead-yet}{\coordname}\sseq@break}
    \ifsseq@keepchanges\else
        \ifnum\the\numexpr\sseq@obj{class.\coord[\sseq@obj{class.\coord.num}].page}<\sseq@thepagecount\else
            \edef\sseq@temp{\sseq@obj{class.\coord[\sseq@obj{class.\coord.num}].page}}
            \sseq@error@xxx{replaceclass-no-effect-on-this-page}{\coord}{\sseq@temp}{\the\numexpr\sseq@temp+\@ne}% warning
            \sseq@breakfifi
        \fi
    \fi
    \let\partialcoord\coord@partial
    \ifx\sseq@class@tag\pgfutil@empty\else
        \sseq@obj@ifdef{partcoord.(\partialcoord).tag.\sseq@class@tagprefix\sseq@class@tag}{%
            \sseq@error@nn{class-tag-already-defined}{\sseq@class@tagprefix\sseq@class@tag}{\sseq@partialcoord}% could be warning
        }{}%
        \sseq@cleanup@obj{partcoord.(\partialcoord).tag.\sseq@class@tagprefix\sseq@class@tag}%
        \sseq@obj@xdef{partcoord.(\partialcoord).tag.\sseq@class@tagprefix\sseq@class@tag}{\nodenum}%
    \fi
    \ifx\sseq@class@name\pgfutil@empty\else
        \sseq@cleanup@obj{class.\classname.name}
        \sseq@cleanup@obj{class.namedclass.\sseq@class@name}
        \sseq@namehandler
        \sseq@obj@xdef{class.\classname.name}{\sseq@class@name}
        \sseq@obj@xdef{class.namedclass.\sseq@class@name}{\coordnopar}
    \fi
    \ifcsname sseq@class@showname\endcsname
        \sseq@eval{\@nx\sseq@handleclassquotes{\sseq@obj{class.\classname.name}}{\sseq@class@showname}}
    \fi
    \sseq@cleanup@obj{class.\coord.num}
    \sseq@obj@inc{class.\coord.num}
    \let\partialcoord\coord@partial
    \ifx\sseq@class@tag\pgfutil@empty\else
        \sseq@obj@ifdef{partialcoord.(\partialcoord).tag.\sseq@class@tagprefix\sseq@class@tag}{%
            \sseq@error@nn{class-tag-already-defined}{\sseq@class@tagprefix\sseq@class@tag}{\sseq@partialcoord}% could be warning
        }{}%
        \sseq@obj@xdef{partialcoord.(\partialcoord).tag.\sseq@class@tagprefix\sseq@class@tag}{\nodenum}%
    \fi
    \edef\classname{class.\coord[\sseq@obj{class.\coord.num}]}
%
    \sseq@obj@xodef{\classname.familylist}{\sseq@familylist}
    \sseq@obj@xdef{\classname.page}{\sseq@infinity}
    \sseq@obj@xodef{\classname.options}{\sseq@savedoptioncode}
    \ifx\sseq@classnodetext\pgfutil@empty\else
        \sseq@obj@xodef{\classname.nodetext}{\sseq@classnodetext}
    \fi
    \ifx\sseq@classnodetextoptions\pgfutil@empty\else
        \sseq@obj@xodef{\classname.nodetext.options}{\sseq@classnodetextoptions}
    \fi
    \sseq@obj@xodef{\classname.labelnodes}{\sseq@classlabelnodes}
    \sseq@obj@gdef{\classname.dependencies}{}
    \ifsseq@needstikz
        \sseq@obj@xdef{\classname.needstikz}{\sseq@needstikztrue}
    \fi
    \@ifundefined{sseq@tempoffset}{}{\tikz@scan@one@point\sseq@setoffset\sseq@tempoffset}
%
    \sseq@breakpoint
    \endgroup
}
\def\sseq@class@draw@ifpage#1{
    \bgroup
    \sseq@obj@pagetogen{#1}{\sseq@thepage} % sets \sseq@gen
    \ifnum\sseq@gen>\m@ne % -1 means no class to print
        \ifnum\sseq@obj{#1[\sseq@gen].page}=\sseq@thepagecount
            \sseq@thispagetrue
        \fi
        \sseq@drawtrue
        \sseq@obj{#1[\sseq@gen].familylist}
        \ifsseq@draw
            \sseq@gsetthiscall{class #1}
            \sseq@class@drawnode{#1[\sseq@gen]}
        \fi
    \fi
    \egroup
}

% #1 -- class
% #2 -- page
% If there is a class #1 drawn on page #2, set \sseq@gen to be the generation of that class, otherwise set \sseq@gen to be -1.
\def\sseq@obj@pagetogen#1#2{
    \bgroup
        \def\sseq@gen{-1}
        \sseq@tempcount=\m@ne
        \sseq@tempcountb=\sseq@obj{#1.num}\relax
        \loop
            \advance\sseq@tempcount\@ne
            \ifnum\sseq@obj{#1[\the\sseq@tempcount].page}<#2\relax\else
                \edef\sseq@gen{\the\sseq@tempcount}
                \sseq@tempcount=\sseq@tempcountb\relax
            \fi
        \ifnum\sseq@tempcount<\sseq@tempcountb\repeat
        \sseq@smuggle@macro\sseq@gen
    \egroup
}
%%%
%%% \kill, \d and \doptions
%%%

% Because \kill doesn't take options, there's no need to use \sseq@DeclareDocumentCommand for the outer command.
% Rather than use \sseq@DeclareDocumentCommand with no arguments, we just do the normal stuff that \sseq@DeclareDocumentCommand would do,
% Then call \sseq@d@grabpage, the command that is responsible for handling the strange syntax of the page argument for \d.
\protected\def\sseq@kill{%
    \begingroup
    \sseq@setinputline % These two lines normally would happen inside of \sseq@DeclareDocumentCommand.
    \sseq@loadinputline
    \sseq@d@grabpage\sseq@kill@
}


% Now we need to parse a potential coordinate.
\sseq@DeclareDocumentCommandAs\sseq@kill@\kill {d()} {
    \sseq@xsetthiscall{\string\kill\unexpanded\@xp{\sseq@dpage}\IfNoValueF{#1}{(#1)}}%
    \sseq@parsecoord\coordinate{\IfNoValueTF{#1}{\lastclass}{#1}}{}%
    \sseq@ifintexpr{\sseq@dpage}{%
        \sseq@tempcount=\numexpr\sseq@dpage\relax % this is a convenient way to get rid of spaces because #3 has to be a number
    }{%
        \sseq@error@x{d-invalid-page}{\unexpanded\@xp{\sseq@dpage}}\sseq@break%
    }%
    \sseq@eval{\@nx\sseq@d@setpageminrec{class.\coordinate}{\the\sseq@tempcount}{coordinate}}%  automatically handles \sseq@cleanup@obj
    \sseq@breakpoint
    \endgroup
}

% \sseq@d@grabpage is defined in sseqparsers.
\sseq@DeclareDocumentCommand\d{o}{%
    \begingroup
    \sseq@loadinputline
    \def\sseq@dtype{d}
    \def\sseq@d@theoptions{#1}
    \sseq@d@grabpage\sseq@d@
}

\sseq@DeclareDocumentCommand\doptions{o}{%
    \begingroup
    \sseq@loadinputline
    \def\sseq@dtype{doptions}
    \def\sseq@d@theoptions{#1}
    \sseq@d@grabpage\sseq@d@
}

\def\sseq@d@{
    \ifsseq@tempif
        \csname sseq@\sseq@dtype @grabcoord\@xp\endcsname
    \else
        \sseq@eval{\@nx\sseq@d@main{\sseq@dtype}{\unexpanded\@xp{\sseq@d@theoptions}}{\sseq@dpage}{\sseq@NoValue}{\sseq@NoValue}}
        \sseq@breakpoint
        \endgroup
    \fi
}

\sseq@DeclareDocumentCommandAs\sseq@d@grabcoord\d{d()d()}{
    \sseq@eval{\@nx\sseq@d@main{\sseq@dtype}{\unexpanded\@xp{\sseq@d@theoptions}}{\sseq@dpage}}{#1}{#2}
    \sseq@breakpoint
    \endgroup
}

\sseq@DeclareDocumentCommandAs\sseq@doptions@grabcoord\doptions{d()d()}{
    \sseq@eval{\@nx\sseq@d@main{\sseq@dtype}{\unexpanded\@xp{\sseq@d@theoptions}}{\sseq@dpage}}{#1}{#2}
    \sseq@breakpoint
    \endgroup
}


\def\sseq@d@main#1#2#3#4#5{%
    \sseq@options@firstpassmode
    \def\sseq@edgetype{differential}
    \the\sseq@sseqstyle
    \the\sseq@edgestyle
    \the\sseq@differentialstyle
    \the\sseq@scope@toks
    \sseq@options@normalmode
    \IfNoValueF{#2}{
        \sseq@processoptions{differential}{#2}% Outputs result into \sseq@savedoptioncode
    }
    \sseq@xsetthiscall{\@xp\string\csname#1\endcsname\IfNoValueF{#2}{\unexpanded{[#2]}}\unexpanded{#3}\IfNoValueF{#4}{\unexpanded{(#4)}}\IfNoValueF{#5}{\unexpanded{(#5)}}}%
    \sseq@ifintexpr{#3}{%
        \sseq@tempcount=\numexpr#3\relax % this is a convenient way to get rid of spaces because #3 has to be a number
    }{%
        \sseq@error@n{d-invalid-page}{#3}\sseq@break%
    }%
    \sseq@eval{\@nx\sseq@dparsecoord{\the\sseq@tempcount}{\IfNoValueTF{#4}{\lastclass0}{\unexpanded{#4}}}{\unexpanded{#5}}}
    \sseq@eval{\unexpanded{\csname sseq@d@#1\endcsname{#2}}{\the\sseq@tempcount}}%
    \sseq@breakpoint
}

% #1 -- options
% #2 -- page
% #3 -- source coordinate
\def\sseq@d@d#1#2{%%
    \sseq@d@setpageminrec{class.\source}{#2}{source}%  automatically handles \sseq@cleanup@obj
    \sseq@d@setpageminrec{class.\target}{#2}{target}
%
    \ifsseq@drawdifferential % TODO: probably should get rid of invisible now that we have \kill...
        \sseq@obj@xodef{d.#2\source\target.familylist}{\sseq@familylist}
        \sseq@obj@gdef{d.#2\source\target.page}{#2}
        \sseq@obj@xdef{d.#2\source\target.source}{\source}
        \sseq@obj@xdef{d.#2\source\target.target}{\target}
        \sseq@obj@xdef{d.#2\source\target.options}{\ifsseq@needstikz\@nx\sseq@needstikztrue\fi\unexpanded\@xp{\sseq@savedoptioncode}}
%
        \sseq@savedpaths@adddifferential{d.#2\source\target}%
    \else
        \sseq@obj@gdef{d.#2\source\target.invisible}{}
    \fi
}
\def\sseq@d@doptions#1#2{%
    \sseq@obj@ifundef{d.#2\source\target.options}{\sseq@error@n{cmdoptions-feature-doesn't-exist}{differential}\sseq@break}{}%
    \sseq@obj@ifdef{d.#2\source\target.invisible}{\sseq@error{doptions-invisible}\sseq@break}
    \ifsseq@keepchanges\else
        \@ifundefined{sseq@thepage}{}{
            \ifnum#2=\sseq@thepagecount\else\ifnum\sseq@thepagecount=\z@\else
                \sseq@error@n{doptions-no-effect-on-this-page}{#2}\sseq@breakfififi
            \fi\fi
        }%
    \fi
%
    \sseq@cleanup@obj{d.#2\source\target.familylist}
    \sseq@obj@xoaddto{d.#2\source\target.familylist}{\sseq@familylist}
    \sseq@cleanup@obj{d.#2\source\target.options}
    \sseq@obj@xoaddto{d.#2\source\target.options}{\sseq@savedoptioncode}
}

% Some helper macros:
% #1 -- class name for \sseq@obj
% #2 -- page
% #3 -- source or target for error handling purposes.
\def\sseq@d@setpageminrec#1#2#3{% not really recursive.
    \sseq@cleanup@obj{#1[\sseq@obj{#1.num}].page}
    \sseq@cleanup@obj{#1[\sseq@obj{#1.num}].deathinfo}
    \sseq@cleanup@obj{#1[\sseq@obj{#1.num}].deathline}
    \sseq@d@setpageclass{#1}{#2}{#3}
    \sseq@obj@map{#1[\sseq@obj{#1.num}].dependencies}{
        \sseq@d@setpagemin{##1}{#2} % automatically handles cleanup
    }
}

% #1 -- class name for \sseq@obj
% #2 -- page
% #3 -- source or target for error handling purposes.
\def\sseq@d@setpageclass#1#2#3{
    \edef\sseq@classgenname{#1[\sseq@obj{#1.num}]}
    % If the class is dead, then it's only not an error if it died on the current page
    \ifnum\sseq@obj{\sseq@classgenname.page}<\sseq@infinitycount
        % If the page it died on is greater than this one, hit-wrong-order error
        \ifnum\sseq@obj{\sseq@classgenname.page}>#2\relax
            \let\sseq@deadclass@genname\sseq@classgenname
            \sseq@preparestacktrace
            \sseq@error@xxxx{d-hit-wrong-order}{#3}{\csname #3name\endcsname}{#2}{\the\sseq@temptoks}
            \sseq@breakfifi
        \fi
        % If the page it died on is less than this one, class-already-hit error
        \ifnum\sseq@obj{\sseq@classgenname.page}<#2\relax
            \let\sseq@deadclass@genname\sseq@classgenname
            \sseq@preparestacktrace
            \sseq@error@xxxx{d-class-already-hit}{#3}{\csname #3name\endcsname}{#2}{\the\sseq@temptoks}
            \sseq@breakfifi
        \fi
    \fi
    % If the class has been replaced before, that's only okay if it was replaced on a previous page.
    \ifnum\sseq@obj{#1.num}>\z@
        \sseq@tempcount=\sseq@obj{#1.num}\relax
        \advance\sseq@tempcount\m@ne
        \ifnum\sseq@obj{#1[\the\sseq@tempcount].page}<#2\relax\else
            \ifnum#2=\sseq@obj{#1[\the\sseq@tempcount].page}
                 \sseq@error@xxxx{d-hit-same-page-replaceclass}{#3}{\csname #3name\endcsname}{#2}{\sseq@obj{#1[\the\sseq@tempcount].deathinfo}}
                 \sseq@breakfififi
            \else
                \edef\sseq@deadclass@genname{#1[\the\numexpr\sseq@obj{#1.num}-\@ne]}
                \sseq@preparestacktrace
                \sseq@error@xxxx{d-hit-wrong-order}{#3}{\csname #3name\endcsname}{#2}{\the\sseq@temptoks}
                \sseq@breakfififi
            \fi
        \fi
    \fi
    \sseq@obj@xdef{\sseq@classgenname.page}{#2}
    \sseq@obj@xdef{\sseq@classgenname.deathinfo}{#3 of differential \sseq@thiscall\space on page #2}
    \bgroup
    \def\foreach{\@nx\@nx\@nx\foreach}
    \protected\def\\{}
    \let\sseq@possibleperiod\empty
    \let\sseq@possibleperiodb\empty
    \sseq@obj@xdef{\sseq@classgenname.deathlocation}{%
        on input line \sseq@inputline\sseq@error@inforeachloop
    }
    \sseq@obj@xdef{\sseq@classgenname.deathline}{\sseq@inputline}
    \sseq@tempiffalse
    \edef\sseq@temp{\the\sseq@foreachcall}
    \ifx\sseq@temp\pgfutil@empty\else\sseq@tempiftrue\fi
    \ifx\sseq@restofstacktrace\pgfutil@empty\else\sseq@tempiftrue\fi
    \sseq@obj@xdef{\sseq@classgenname.deathstacktrace}{%
        \ifsseq@tempif
            \unexpanded{\\\\}Stack trace for differential \sseq@thiscall\ that killed #3:\unexpanded{\\}%
            \ifx\sseq@temp\sseq@empty\else\unexpanded{\ \ \ }\@nx\sseq@stack@unexpanded{\unexpanded\@xp{\the\sseq@foreachcall}}\fi
            \unexpanded\@xp{\sseq@restofstacktrace}%
        \fi
    }
    \egroup
}

\def\sseq@preparestacktrace{
    \def\foreach{\@nx\foreach}
    \edef\sseq@temp@i{\unexpanded\@xpthree{\sseq@obj{\sseq@deadclass@genname.deathline}}}
    \edef\sseq@temp@ii{\sseq@inputline}
    \ifx\sseq@temp@i\sseq@temp@ii
        \sseq@temptoks\@xpthree{\sseq@obj{\sseq@deadclass@genname.deathinfo} (in the same \foreach loop)}
    \else
        \protected\def\\{}
        \sseq@eval{\sseq@temptoks{\unexpanded\@xpthree{\sseq@obj{\sseq@deadclass@genname.deathinfo}}\space(\sseq@obj{\sseq@classgenname.deathlocation})}}
        \sseq@e@addto@macro\sseq@errorinfo{\unexpanded\@xpthree{\sseq@obj{\sseq@deadclass@genname.deathstacktrace}}}
    \fi
}

\def\sseq@d@setpagemin#1#2{
    \ifnum\sseq@obj{#1.page}>#2\relax
        \sseq@cleanup@obj{#1.page}
        \sseq@obj@xdef{#1.page}{#2}
    \fi
}
\def\sseq@differential@draw@ifpage#1{
    \sseq@drawtrue
    \sseq@obj{#1.familylist}
    \ifsseq@draw
        \ifnum\sseq@thepagecount=\z@
            \sseq@drawdifferential{#1}
        \else
            \ifnum\sseq@obj{#1.page}>\sseq@thepagemm\relax
                \ifnum\sseq@obj{#1.page}<\sseq@thepagemaxpp\relax
                    \sseq@drawdifferential{#1}
                \fi
            \fi
        \fi
    \fi
}
\def\sseq@drawdifferential#1{
    \bgroup
    \sseq@eval{
        \@nx\sseq@drawedge\sseq@obj{#1.source}\sseq@obj{#1.target}{differential}{%
            \@xptwo\@nx\sseq@obj{#1.options}
        }
    }
    \egroup
}
%%%
%%% \structline and \structlineoptions
%%%

\sseq@DeclareDocumentCommand\structline{od()}{%
    \begingroup
    \sseq@loadinputline
    \IfNoValueTF{#2}{
        \sseq@structline@main{structline}{#1}{#2}{#2}
    }{
        \def\next{\sseq@structline@@{structline}{#1}{#2}}%
        \@xptwo\next\sseq@trimleadingspaces
    }
}
\sseq@DeclareDocumentCommand\structlineoptions{od()}{%
    \begingroup
    \sseq@loadinputline
    \IfNoValueTF{#2}{
        \sseq@structline@main{structlineoptions}{#1}{#2}{#2}
    }{
        \def\next{\sseq@structlineoptions@@{structlineoptions}{#1}{#2}}%
        \@xptwo\next\sseq@trimleadingspaces
    }
}

\sseq@DeclareDocumentCommandAs\sseq@structline@@\structline{mmmd()}{%
    \sseq@structline@main{#1}{#2}{#3}{#4}
}

\sseq@DeclareDocumentCommandAs\sseq@structlineoptions@@\structlineoptions{mmmd()}{%
    \sseq@structline@main{#1}{#2}{#3}{#4}
}

\def\sseq@structline@main#1#2#3#4{
    \sseq@xsetthiscall{\string\structline\IfNoValueF{#2}{\unexpanded{[#2]}}\IfNoValueF{#3}{\unexpanded{(#3)}\IfNoValueF{#4}{\unexpanded{(#4)}}}}%
    \def\sseq@edgetype{structline}
    \sseq@options@firstpassmode
    \the\sseq@sseqstyle
    \the\sseq@edgestyle
    \the\sseq@structlinestyle
    \the\sseq@scope@toks
    \sseq@options@normalmode
    \IfNoValueF{#2}{%
        \sseq@processoptions{struct line}{#2}% Outputs result into \sseq@savedoptioncode
    }%
    \def\sourcename{#3}%
    \def\targetname{#4}%
    \IfNoValueTF{#3}{%
        \def\sourcename{\lastclass1}%
        \def\targetname{\lastclass}%
    }{%
        \IfNoValueT{#4}{%
            \def\targetname{\lastclass}%
        }%
    }%
    \sseq@parsecoord\source{\sourcename}{source of structure line}%
    \sseq@parsecoord\target{\targetname}{target of structure line}%
    \ifnum\xsource>\xtarget\relax
        \sseq@parsedcoords@swap\source\target
    \else
        \ifnum\xsource=\xtarget\relax
            \ifnum\ysource>\ytarget\relax
                \sseq@parsedcoords@swap\source\target
            \else
                \ifnum\ysource=\ytarget\relax
                    \ifnum\nsource>\ntarget\relax
                        \sseq@parsedcoords@swap\source\target
                    \else
                        \ifnum\nsource=\ntarget\relax
                            \sseq@error{structline-source-target-equal}%
                            \@xp\sseq@break\romannumeral-`0
                        \fi
                    \fi
                \fi
            \fi
        \fi
    \fi
    \@ifundefined{sseq@structline@page}{%this is repeated in circleclasses
        \def\sseq@pagemin{0}
        \ifnum\sseq@obj{class.\source.num}>\z@
            \edef\sseq@pagemin{\sseq@obj{class.\source[\the\numexpr\sseq@obj{class.\source.num}-1].page}}
        \fi
        \ifnum\sseq@obj{class.\target.num}>\z@
            \edef\sseq@test{\sseq@obj{class.\target[\the\numexpr\sseq@obj{class.\target.num}-1].page}}
            \ifnum\sseq@test>\sseq@pagemin\relax
                \let\sseq@pagemin\sseq@test
            \fi
        \fi
    }{\edef\sseq@pagemin{\the\numexpr\sseq@structline@page-\@ne}}
    \csname sseq@structline@#1\endcsname
    \sseq@breakpoint
    \endgroup
}
\def\sseq@structline@structline{
    \@ifundefined{sseq@structline@pagemax}{}{\sseq@error{structline-no-page-max}}% could be warning
    \ifnum\sseq@pagemin<\sseq@obj{class.\source[\sseq@obj{class.\source.num}].page}\relax\else
        \sseq@error{structline-stillborn}
        \sseq@breakfi
    \fi
    \ifnum\sseq@pagemin<\sseq@obj{class.\target[\sseq@obj{class.\target.num}].page}\relax\else % If the \structline is dead before it is born, throw an error
        \sseq@error{structline-stillborn}
        \sseq@breakfi
    \fi
    \sseq@obj@ifdef{structline.\source\target.num}{%
        \ifnum\sseq@obj{structline.\source\target[\sseq@obj{structline.\source\target.num}].page}=\sseq@infinitycount
            \sseq@error@xx{structline-already-exists}{\sourcename}{\targetname}
            \sseq@breakfi
        \fi
        \sseq@cleanup@obj{structline.\source\target.num}
        \sseq@obj@inc{structline.\source\target.num}
    }{
        \sseq@cleanup@obj{structline.\source\target.num}
        \sseq@obj@gdef{structline.\source\target.num}{0}
        \sseq@savedpaths@addstructline{structline.\source\target}
    }
    \edef\sseq@structlinename{structline.\source\target[\sseq@obj{structline.\source\target.num}]}
    % The following cleanup call is needed to prevent a glitch where if a \structline is made inside of a temporary sseqpage environment, then later attempts
    % to put a structline throw a "structline already exists" error
    \sseq@cleanup@obj{\sseq@structlinename.page}
    \sseq@obj@xdef{\sseq@structlinename.page}{\sseq@infinity}
    \sseq@obj@xdef{\sseq@structlinename.pagemin}{\sseq@pagemin}
    \sseq@obj@xdef{\sseq@structlinename.source}{\source}
    \sseq@obj@xdef{\sseq@structlinename.target}{\target}
    \sseq@class@adddependence{class.\source}{\sseq@structlinename}% If the source or target is already dead, this will set structlinename.page
    \sseq@class@adddependence{class.\target}{\sseq@structlinename}
    \sseq@obj@xodef{\sseq@structlinename.familylist}{\sseq@familylist}
    \sseq@cleanup@obj{\sseq@structlinename.options}
    \sseq@obj@xdef{\sseq@structlinename.options}{\ifsseq@needstikz\@nx\sseq@needstikztrue\fi\unexpanded\@xp{\sseq@savedoptioncode}}
}
\def\sseq@structline@structlineoptions{
    \sseq@obj@ifundef{structline.\source\target.num}{
        \sseq@error@n{cmdoptions-feature-doesn't-exist}{\structline}
        \sseq@break
    }{}
    \@ifundefined{sseq@structline@page}{
        \edef\sseq@gen{\sseq@obj{structline.\source\target.num}}
        \let\sseq@genmax\sseq@gen
    }{
        \sseq@structline@getgen{\sseq@structline@page}
        \@ifundefined{sseq@structline@pagemax}{%
            % If no max is present, apply options to all pages starting at specified minimum
            \edef\sseq@genmax{\sseq@obj{structline.\source\target.num}}%
            \let\sseq@structline@pagemax\sseq@infinity
        }{
            \let\sseq@genmin\sseq@gen
            \sseq@structline@getgen{\sseq@structline@pagemax}
            \let\sseq@genmax\sseq@gen
            \let\sseq@gen\sseq@genmin
        }
        \edef\sseq@structline@page{\the\numexpr\sseq@structline@page-1}
        \edef\sseq@structline@pagemax{\the\numexpr\sseq@structline@pagemax+1}
    }
    \sseq@tempcount=\sseq@gen\relax
    \advance\sseq@tempcount\m@ne
    \loop
        \advance\sseq@tempcount\@ne
        \edef\sseq@structlinename{structline.\source\target[\the\sseq@tempcount]}
        \sseq@cleanup@obj{\sseq@structlinename.options}
        \sseq@cleanup@obj{\sseq@structlinename.familylist}
        \ifsseq@needstikz
            \sseq@obj@gaddto{\sseq@structlinename.options}{\sseq@needstikztrue}
        \fi
        \ifnum\sseq@tempcount=\sseq@gen\relax
            \@ifundefined{sseq@structline@page}{}{
                \edef\sseq@savedoptioncode{\@nx\ifnum\sseq@thepagecount>\sseq@structline@page\relax\unexpanded\@xp{\sseq@savedoptioncode}\@nx\fi}
                \edef\sseq@familylist{\@nx\ifnum\sseq@thepagecount>\sseq@structline@page\relax\unexpanded\@xp{\sseq@familylist}\@nx\fi}
            }
        \fi
        \ifnum\sseq@tempcount=\sseq@genmax\relax
            \@ifundefined{sseq@structline@pagemax}{}{
                \edef\sseq@savedoptioncode{\@nx\ifnum\sseq@thepagecount<\sseq@structline@pagemax\relax\unexpanded\@xp{\sseq@savedoptioncode}\@nx\fi}
                \edef\sseq@familylist{\@nx\ifnum\sseq@thepagecount<\sseq@structline@pagemax\relax\unexpanded\@xp{\sseq@familylist}\@nx\fi}
            }
        \fi
        \sseq@obj@xoaddto{\sseq@structlinename.options}{\sseq@savedoptioncode}
        \sseq@obj@xoaddto{\sseq@structlinename.familylist}{\sseq@familylist}
    \ifnum\sseq@tempcount<\sseq@genmax\repeat
}

\def\sseq@structline@copy#1#2{
    \begingroup
    \sseq@structline@copy@decomposename#2
    \let\sseq@thename@saved\sseq@thename
    \def\sseq@pagemin{0}
    \sseq@parsecoord\source{\sourcename}{source of structure line}%
    \sseq@parsecoord\target{\targetname}{target of structure line}
    \def\sseq@thename{#1}
    \sseq@lettoobj\sseq@familylist{#2.familylist}
    \sseq@lettoobj\sseq@savedoptioncode{#2.options}
    \let\sseq@thename\sseq@thename@saved
    \sseq@structline@structline
    \endgroup
}

\def\sseq@structline@copy@decomposename#1.(#2,#3,#4)(#5,#6,#7){
    \def\sourcename{#2,#3,copiedfrom#4}
    \def\targetname{#5,#6,copiedfrom#7}
}

\def\sseq@structline@getgen#1{
    \bgroup
        \sseq@tempcount=\sseq@obj{structline.\source\target.num}\relax
        \ifnum\sseq@tempcount>\z@
            \ifnum#1<\z@\relax
                \@ifundefined{sseq@thepage}{
                    \ifnum\sseq@obj{structline.\source\target[\the\sseq@tempcount].page}<\sseq@infinitycount
                        \sseq@tempcount=\sseq@obj{structline.\source\target[\the\sseq@tempcount].page}\relax
                    \else
                        \advance\sseq@tempcount\m@ne
                        \sseq@tempcount=\sseq@obj{structline.\source\target[\the\sseq@tempcount].page}\relax
                        \advance\sseq@tempcount\@ne
                    \fi
                }{
                    \sseq@tempcount=\sseq@thepagecount
                }
                \advance\sseq@tempcount\@ne
                \advance\sseq@tempcount#1\relax
            \fi
            \sseq@eval{\unexpanded{\sseq@obj@pagetogen{structline.\source\target}}{\the\sseq@tempcount}} % stores in \sseq@gen
        \else
            \def\sseq@gen{0}
        \fi
        \sseq@smuggle@macro\sseq@gen
    \egroup
}
\def\sseq@class@adddependence#1#2{
    \edef\sseq@temp{\sseq@obj{#1.num}}
    \sseq@d@setpagemin{#2}{\sseq@obj{#1[\sseq@temp].page}}% automatically handles cleanup
    \sseq@cleanup@obj{#1[\sseq@temp].dependencies}
    \sseq@obj@xaddto{#1[\sseq@temp].dependencies}{\@nx\\{#2}}
}
\def\sseq@structline@draw@ifpage#1{
    \bgroup
    \ifnum\sseq@thepagecount=\z@
        \ifnum\sseq@obj{#1[0].pagemin}=\z@
            \sseq@drawtrue
            \sseq@obj{#1[0].familylist}
            \ifsseq@draw
                \sseq@structline@draw{#1[0]}
            \fi
        \fi
    \else
        \sseq@obj@pagetogen{#1}{\sseq@thepage} % sets \sseq@gen
        \ifnum\sseq@gen<\z@\else % negative sseq@gen means don't draw it
            \ifnum\sseq@obj{#1[\sseq@gen].page}=\sseq@thepagecount
                \sseq@thispagetrue
            \fi
            \ifnum\sseq@obj{#1[\sseq@gen].pagemin}<\sseq@thepagecount
                \sseq@drawtrue
                \sseq@obj{#1[\sseq@gen].familylist}
                \ifsseq@draw
                    \sseq@structline@draw{#1[\sseq@gen]}
                \fi
            \fi
        \fi
    \fi
    \egroup
}
\def\sseq@structline@draw#1{\sseq@eval{\@nx\sseq@drawedge\sseq@obj{#1.source}\sseq@obj{#1.target}{structline}{\@nx\sseq@obj{#1.options}}}}




\newcount\sseq@circleclassuid
\sseq@DeclareDocumentCommand\circleclasses{or()r()}{
    \begingroup
    \sseq@loadinputline
    \def\sseq@keysfortikzprim{}
    \IfNoValueF{#1}{
        \sseq@processoptions{fit}{#1}
    }
    \sseq@parsecoord\coorda{#2}{first \@nx\circleclass class}
    \sseq@parsecoord\coordb{#3}{second \@nx\circleclass class}
%
%
    \ifsseq@keepchanges
        \@ifundefined{sseq@fit@page}{% also in structline@main. TODO: refactor these chunks together?
            \def\sseq@pagemin{0}
            \ifnum\sseq@obj{class.\coorda.num}>\z@
                \edef\sseq@pagemin{\sseq@obj{class.\coorda[\the\numexpr\sseq@obj{class.\coorda.num}-\@ne].page}}
            \fi
            \ifnum\sseq@obj{class.\coordb.num}>\z@
                \edef\sseq@test{\sseq@obj{class.\coordb[\the\numexpr\sseq@obj{class.\coordb.num}-\@ne].page}}
                \ifnum\sseq@test>\sseq@pagemin\relax
                    \let\sseq@pagemin\sseq@test
                \fi
            \fi
        }{\edef\sseq@pagemin{\the\numexpr\sseq@fit@page-\@ne}}
        \@ifundefined{sseq@fit@pagemax}{\let\sseq@pagemax\sseq@infinity}{\let\sseq@pagemax\sseq@fit@pagemax}
    \else % Put it on the current page no matter what inside sseqpage
        \let\sseq@pagemax\sseq@infinity\def\sseq@pagemin{-1}
    \fi
%
%
    \global\advance\sseq@circleclassuid\@ne
    \edef\circleclassobjname{circleclass(\coordanopar)(\coordbnopar)\the\sseq@circleclassuid}
%
%
    \sseq@obj@xodef{\circleclassobjname.options}{\sseq@savedoptioncode}
    \sseq@obj@xodef{\circleclassobjname.familylist}{\sseq@familylist}
    \sseq@obj@xodef{\circleclassobjname.tikzprimoptions}{\sseq@keysfortikzprim}
    \sseq@obj@xdef{\circleclassobjname.fitnodes}{\@nx\tikz@lib@fit{(sseq{\coordanopar})(sseq{\coordbnopar})\sseq@fitalso}}
    %\edef\temp{\@nx\sseq@circleclass@draw{\coorda@internalname}{\coordb@internalname}{\unexpanded\@xp{\sseq@savedoptioncode}}}
    \sseq@savedpaths@xadd{\@nx\sseq@circleclass@object{\circleclassobjname}{\coordanopar}{\coordbnopar}}
    \sseq@obj@xdef{\circleclassobjname.page}{\sseq@pagemax}
    \sseq@obj@xdef{\circleclassobjname.pagemin}{\sseq@pagemin}
    \@ifundefined{sseq@fit@pagemax}{
        \ifsseq@keepchanges
            \sseq@class@adddependence{class.\coorda}{\circleclassobjname}% If the source or target is already dead, this will set structlinename.page
            \sseq@class@adddependence{class.\coordb}{\circleclassobjname}
        \fi
    }{}
    \sseq@breakpoint
    \endgroup
}

\ifsseq@patchfit\else
    \sseq@DeclareDocumentCommand\circleclasses{or()r()}{%
        \sseq@warning{circleclasses-not-provided}
    }
\fi

\def\sseq@circleclass@draw@ifpage#1#2#3{
    \sseq@drawtrue
    \sseq@obj{#1.familylist}
    \ifsseq@draw
        \ifnum\sseq@thepagecount<\numexpr\sseq@obj{#1.page}+\@ne
            \ifnum\sseq@obj{#1.page}=\sseq@thepagecount
                \sseq@thispagetrue
            \fi
            \ifnum\sseq@obj{#1.pagemin}<\sseq@thepagecount
                \sseq@circleclass@draw{#1}{#2}{#3}
            \fi
        \fi
    \fi
}

%%
%%  Labels
%%
%%    This code is called from the key value handler, but it has more to do with the code in this file so I am putting it here. Search in sseqkeys.code.tex for "Labels"
%%

\def\sseq@processlabels@default{%
    \global\let\sseq@classnodetext\pgfutil@empty
    \global\let\sseq@classnodetextoptions\pgfutil@empty
    \global\let\sseq@classlabelnodes\pgfutil@empty
}
\let\sseq@processlabels\sseq@processlabels@default

% I can't remember why we need to defer this. If at some point I figure this out, I should explain why here and in the definition of \class...
\def\sseq@handleclassquotes@inner#1#2{%
    \begingroup\pgfscope
    \def\sseq@classquotes@smugglehook{}
    \def\sseq@classquotes@smugglehook@inner{}
    \sseq@options@firstpassmode
    \sseq@thesseqstyle
    \sseq@thelabelstyle
    \sseq@theclasslabelstyle
    \the\sseq@scope@toks
    \sseq@options@normalmode
    \sseq@processoptions{class/label}{#2}%
    \sseq@classlabel@handler{#1}% labeltextfn is a private handler, classlabeltextfn is exposed via "class label handler"
    \ifsseq@classlabel
        \sseq@options@firstpassmode
        \the\sseq@outerclasslabelstyle
        \sseq@x@addto@macro\sseq@classlabelnodes{%
            \@nx\sseq@drawlabel{%
                \sseq@maybemathswitch\unexpanded\@xp{\processedlabel}\sseq@maybemathswitch}%
                {% This is set up to prevent a bug where \classoptions and class label style don't work correctly together
                    \the\sseq@sseqstyle\@nx\the\sseq@sseqstyle@page
                    \the\sseq@labelstyle\@nx\the\sseq@labelstyle@page
                    \the\sseq@classlabelstyle\@nx\the\sseq@classlabelstyle@page
                    \sseq@theouterclasslabelstyle
                    \@nx\sseq@collections@labels@hook
                    \@nx\sseq@collections@classlabels@hook
                    \@nx\sseq@collections@outerclasslabels@hook
                    \unexpanded\@xp{\sseq@savedoptioncode}%
                }%
        }%
    \else
        \let\sseq@classquotes@smugglehook\sseq@classquotes@smugglehook@inner
        \sseq@options@firstpassmode
        \the\sseq@innerclasslabelstyle
        \sseq@options@normalmode
        \sseq@protectedeval{\@nx\pgfqkeys{/sseqpages/class/label}{#2}}%
        \sseq@protected@xdef\sseq@classnodetext{\sseq@maybemathswitch\processedlabel\sseq@maybemathswitch}%
        \xdef\sseq@classnodetextoptions{%
            %\unexpanded{\def\pgfkeysdefaultpath{/sseqpages/class/node texts/}}%
            \sseq@thelabelstyle\sseq@theclasslabelstyle\sseq@theinnerclasslabelstyle
            \@nx\sseq@collections@labels@hook
            \@nx\sseq@collections@classlabels@hook
            \@nx\sseq@collections@innerclasslabels@hook
            \unexpanded\@xp{\sseq@savedoptioncode}%
        }%
    \fi
    \@xp\endpgfscope\@xp\endgroup
    \sseq@classquotes@smugglehook
}

\def\sseq@handleedgequotes#1#2{% We use this via \let\tikz@quotes@as\sseq@handleedgequotes
    /utils/exec={% so it's called inside \pgfkeysalso{stuff}. Need to surround with /utils/exec={} to get out.
        \let\sseq@savedoptioncode@save\sseq@savedoptioncode
        \sseq@processoptions{label}{#2}
        \sseq@needstikztrue
        \let\sseq@temp\sseq@savedoptioncode
        \let\sseq@savedoptioncode\sseq@savedoptioncode@save
        \sseq@e@addto@options{
            \@nx\tikzset{
                every to/.append style={
                    edge node={
                        node [/sseqpages/label,
                             /utils/exec={
                                \let\@nx\sseq@tikz@transform@save\@nx\tikz@transform
                                \let\@nx\tikz@transform\relax
                                \@nx\sseq@thelabelstyle
                                \@nx\sseq@collections@labels@hook
                                \@nx\sseq@theedgelabelstyle
                                \@nx\sseq@collections@edgelabels@hook
                                \@xp\@nx\csname sseq@the\sseq@edgetype labelstyle\endcsname
                                \@xp\@nx\csname sseq@collections@\sseq@edgetype labels@hook\endcsname
                                \unexpanded\@xp{\sseq@temp}
                                \ifsseq@rotatelabels\iftikz@fullytransformed\else\@nx\tikz@fullytransformedtrue\@nx\tikz@addtransform{\@nx\pgftransformresetnontranslations\@nx\sseq@globalrotatetransform}\fi\fi
                                 \unexpanded\@xp{\sseq@edgeoptionhook} % this is just to make description work
                                 \let\@nx\tikz@transform\@nx\sseq@tikz@transform@save
                            }
                        ]{\sseq@labeltextfn{#1}}
                   }
                }
            }
        }
    }
}


